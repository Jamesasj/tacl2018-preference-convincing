\section{Introduction}\label{sec:intro}

Arguments are intended to persuade the audience of a particular point of view and 
are an important way for humans to reason about controversial topics~\cite{mercier2011humans}. 
The amount of argumentative text on any chosen topic can, however, overwhelm a reader.
Consider the scale of historical text archives 
and the debates on social media platforms with millions of users.
Automated methods could enable readers to overcome this challenge by identifying high-quality, 
persuasive arguments from different sides of a debate. 

Theoretical approaches for assessing argument quality have proved difficult to apply to everyday arguments~\cite{boudry2015fake}.
However, empirical approaches using machine learning have recently shown success in 
identifying convincing arguments in online discussions~\cite{habernal2016argument}.
\begin{figure}
\textbf{Topic:} ``William Farquhar ought to be honoured as the rightful founder of Singapore". \\
\textbf{Stance:} ``No, it is Raffles!" \\
\textbf{Argument 1:}  
HE HAS A BOSS(RAFFLES) HE HAS TO FOLLOW HIM AND NOT GO ABOUT DOING ANYTHING ELSE... \\
\textbf{Argument 2:} 
Raffles conceived a town plan to remodel Singapore into a modern city. The plan consisted of separate areas for different...
\caption{Example argument pair from an online debate.}
\label{fig:argument_examples}
\end{figure}
Typically, machine learning approaches require examples of arguments paired with judgements of their convincingness.
However, consider the arguments in Figure \ref{fig:argument_examples}:
how does one assign a numerical convincingness score to each argument? 
If the audience considers each argument independently, it is difficult to ensure that the scores for all arguments
remain consistent with their view of convincingness.

An alternative way to judge arguments is to compare them against one another. 
In the case of arguments 1 and 2 in Figure \ref{fig:argument_examples}, we may judge that argument 1 is less convincing due to its writing style, whereas argument 2 
presents evidence in the form of historical events.
Pairwise comparisons such as this are known to place less cognitive burden on human annotators than  
choosing a numerical rating and allow fine-grained sorting of items that is not possible with categorical labels
~\cite{kendall1948rank,kingsley2006preference}.
Relative judgements also avoid the problem that different annotators 
may have biases toward high, low or middling ratings, making their scores hard to compare.

% \todo{the reliance of these models on massive sets of hand-labeled training data. data acquisition bottleneck. Prevalent in NLP: consider wordnet versus euronet for languages --> hard to get non-English. See comment on collobert.
% % Typical neural network approaches have achieved success by using more than 200,000 training data points for each task\cite{}.
% Not really connected to previous paragraph?
% }
In practice, we face a data acquisition bottleneck when encountering new domains or audiences.
For example, machine learning methods such as neural networks typically require datasets with 
many thousands of hand-labelled examples to perform well~\cite{srivastava2014dropout,collobert2011natural}.
One solution is to employ multiple non-specialist annotators at low cost (\emph{crowdsourcing}), 
but this requires quality control techniques to account for errors.
Another source of data are the actions of users of a software application, which can be interpreted as pairwise judgements~\cite{joachims2002optimizing}. For example, when a user clicks on an argument in a list it can be interpreted
 as a preference for the selected argument over more highly-ranked arguments.
However, the resulting pairwise labels are an extremely noisy indication of preference.

% \todo{motivate use of word embeddings as additional input to GP. 
% They have been shown to provided complementary information in applications such as... Mainly because they provided some semantic information not captured by the
% other features.}
% \todo{making software available -- see TACL?}
In this paper, we develop a Bayesian approach to learn from noisy pairwise preferences
based on Gaussian process preference learning (GPPL)~\cite{chu2005preference}.
We model argument convincingness as a function of textual features, including word embeddings,
and develop an inference method for GPPL that scales to realistic dataset sizes using a stochastic variational inference (SVI) ~\cite{hoffman2013stochastic}. % and make our software available at...
Using datasets provided by Habernal and Gurevych~\shortcite{habernal2016argument},
we show that our method outperforms the previous state-of-the-art for ranking arguments by convincingness 
and identifying the most convincing argument in a pair. 
%We also show the benefit of combining linguistic features with word embeddings.
Further experiments show that our Bayesian approach is particularly advantageous with small, noisy datasets, 
and in an active learning set-up.

The rest of the paper is structured as follows.
Section \ref{sec:related} reviews related work on argumentation,
then Section \ref{sec:bayesian} motivates the use of Bayesian methods by discussing their successful applications in NLP.
In Section \ref{sec:pref_learning}, we review preference learning methods and then Section \ref{sec:model}
describes our scalable Gaussian process-based approach.
Section \ref{sec:expts} presents our evaluation, 
comparing our method to the state-of-the art and testing with noisy data and active learning.
Finally, we present conclusions and future work.

\section{Identifying Convincing Arguments}\label{sec:related}

% Recently, Habernal and Gurevych~\shortcite{habernal2016makes} analysed reasons provided by annotators for why one argument is more convincing than another. In this paper we assume that explicit reasons are not provided. 
Lukin et al.~\shortcite{lukin2017argument} demonstrated that an audience's personality and prior stance affect
an argument's persuasiveness, but they were unable to predict belief change to a high degree of accuracy.
Related work has shown how persuasiveness is also affected by the sequence of arguments in a discussion 
~\cite{tan2016winning,rosenfeld2016providing,monteserin2013reinforcement},
but this work focuses on predicting salience of an argument given the state of the debate,
 rather than the qualities of arguments.

Habernal and Gurevych~\shortcite{habernal2016argument} established datasets
containing crowdsourced pairwise judgements of convincingness for arguments taken from online discussions. 
Errors in the crowdsourced data were handled by determining gold labels using the MACE algorithm~\cite{hovy2013learning}.
The gold labels were then used to train SVM and bi-directional long short-term memory (BiLSTM) classifiers to predict pairwise
labels for new arguments. 
The gold labels were also used to construct a directed graph of convincingness, which was input to PageRank 
to produce scores for each argument. 
These scores were then used to train SVM and BiLSTM regression models.
A drawback of such pipeline approaches is that they are prone to error propagation~\cite{chen2016joint},
and consensus algorithms such as MACE require multiple crowdsourced labels for each argument pair, 
which increases annotation costs.

\section{Bayesian Methods for NLP}\label{sec:bayesian}

When faced with a lack of reliable annotated data, 
Bayesian approaches have a number of advantages.
Bayesian inference provides a mathematical framework for combining multiple observations
with prior information. 
Given a model, $M$, and observed data, $D$, we apply Bayes' rule
to obtain a posterior distribution over $M$, which can be used to make predictions 
about unseen data or latent variables:
\begin{equation}
  P(M|D) = \frac{P(D|M)P(M)}{P(D)},
  \label{eq:bayesrule}
\end{equation}
where $P(D|M)$ is the likelihood of the data given $M$, and $P(M)$ is the model prior.
If the dataset is large, the likelihood will dominate the posterior,
but when $D$ is small, the posterior will remain closer to the prior.
%which reduces overfitting.
Rather than learning a posterior distribution over $M$,
neural network training typically selects model parameters that maximise the likelihood, 
so they are prone to overfitting with small datasets, which can reduce performance~\cite{xiong2011bayesian}.

% \todo{link each paragraph to argumentation problem: no trustworthy labels; multiple annotators; ?}
Bayesian methods can be trained using unsupervised or semi-supervised learning
to take advantage of structure in unlabelled data when labelled data is in short supply.
Popular examples in NLP are
Latent Dirichlet Allocation (LDA)~\cite{blei2003latent}, which is used for topic modelling,
and its extension, the hierarchical Dirichlet process (HDP)~\cite{teh2005sharing}, which learns the 
the number of topics than requiring it to be fixed a priori.
Semi-supervised Bayesian learning
has also been used to achieve state-of-the-art results for semantic role labelling~\cite{titov2012bayesian}.

We can combine independent pieces of weak evidence using Bayesian methods through the likelihood.
For instance, a Bayesian network can be used to infer attack relations between arguments by combining votes for acceptable arguments from different people~\cite{kido2017}.
Graphical models can also combine crowdsourced annotations to train a sentiment classifier
without a separate quality control step~\cite{simpson2015language,felt2016semantic}.

Several successful Bayesian approaches in NLP make use of Gaussian processes (GPs), which are 
distributions over functions of input features. 
GPs are nonparametric, meaning they can model highly nonlinear functions by
allowing function complexity to grow with the amount of data~\cite{rasmussen_gaussian_2006}.
They account for model uncertainty when extrapolating from sparse training data
and can be incorporated into larger graphical models.
Example applications include analysing the relationship between a user's impact on Twitter 
and the text features of their tweets~\cite{lampos2014predicting}, 
predicting the level of emotion in text~\cite{beck2014joint},
and estimating the quality of machine translations given source and translated texts~\cite{cohn2013modelling}.
% In scenarios where there is a high cost of obtaining annotations, active learning techniques
% can be used to obtain more informative annotations by making use of the model uncertainty estimates provided by
% Bayesian approaches~\cite{mackay1992information}.
% GPs in NLP:
% State of the art for regression
% I supports very complex non-linear functions
% I elegant model selection
% Now mature enough for use in NLP
% I sparse approximations for large scale inference
% I probabilistic formulation allows incorporation into larger
% graphical models
% I models the prediction uncertainty so that it’s propagated
% through pipelines of probabilistic components

\section{Preference Learning}\label{sec:pref_learning}

% GP-based approach versus Bradley/Terry/Luce/MPM method or Mallows models?
Our aim is to develop a Bayesian method for identifying convincing arguments 
given their features, which can be trained on noisy pairwise labels.
Each label, $i \succ j$, states that an argument, $i$, is more convincing than another argument, $j$. 
This learning task is a form of \emph{preference learning}, which can be addressed in several ways.
% rank or score arguments in terms of convincingness,
% or predict which item in a subset the user finds more convincing. 
%Given a ranking over items, it is possible to determine the pairwise preferences, but 
A simple approach is to use a generic classifier by
obtaining a single feature vector for each pair in the training and test datasets,
either by concatenating the feature vectors of the items in the pair 
or by computing the difference of the two feature vectors, as in SVM-Rank~\cite{joachims2002optimizing}. 
However, this approach does not produce ranked lists of convincing arguments without predicting a large number of pairwise labels, nor give scores of convincingness. %can it quantify the difference in convincingness between arguments. 

Alternatively, we can learn an ordering over arguments directly using Mallows models~\cite{mallows1957non},
which define distributions over list permutations.  
Mallows models can be trained from pairwise preferences 
~\cite{lu2011learning}, but inference is typically costly
since the number of possible permutations is $\mathcal{O}(N^2)$, 
where $N$ is the number of arguments. 
Modelling only the ordering does not allow us to quantify 
the difference between arguments at similar ranks.

To avoid the problems of classifier-based and permutation-based methods, 
we propose to learn a real-valued convincingness function, $f$, that takes argument features as input
and can be used to predict rankings, pairwise labels, or ratings for individual arguments.
There are two well established approaches for mapping pairwise labels to real-valued scores: 
the Bradley-Terry-Plackett-Luce model~\cite{bradley1952rank,luce1959possible,plackett1975analysis}
and the Thurstone-Mosteller model~\cite{thurstone1927law,mosteller2006remarks}.
Based on the latter approach, 
Chu and Ghahramani~\shortcite{chu2005preference} introduced 
Gaussian process preference learning (GPPL), 
a Bayesian model that can tolerate errors in pairwise training labels
and gains the advantages of a GP for learning nonlinear functions from sparse datasets.
% The Gaussian process (GP) preference learning approach of \cite{chu2005preference} resolves inconsistencies between preferences and provides a way to predict rankings or preferences for 
% items for which we have not observed any pairwise comparisons based on the item's features. 
% This model assumes that preferences are noisy, i.e. contain some erroneous labels.
% particularly as the modular nature of inference algorithms such as Gibb's sampling and variational approximation is suited to extending the model to handle different types of feedback that give indications of some underlying preferences. 
However, the inference method proposed by Chu and Ghahramani~\shortcite{chu2005preference} 
has memory and computational costs that scale with $\mathcal{O}(N^3)$,
making it unsuitable for real-world text datasets. 
The next section explains how we use recent developments in inference methods 
to develop scalable Bayesian preference learning for argument convincingness.

%GPs for NLP in other task areas?
