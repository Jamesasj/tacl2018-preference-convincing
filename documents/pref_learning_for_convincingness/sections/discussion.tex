\section{Conclusions and Future Work}

We presented a novel, scalable approach to predicting argument convincingness 
using Bayesian preference learning, and demonstrated how our method outperforms the state-of-the-art.
We showed particularly strong performance with sparse and noisy training data,
as may be found in crowdsourcing or interactive learning scenarios.
%Several related NLP and argumentation problems could benefit from a similar methodology, particularly in an 
%interactive setting where large amounts of clean training data are unavailable. 
%An example is the argument reasoning comprehension task\cite{habernal2017arg}, 
%where annotators select preferred components to complete an argument. 
Future work will evaluate our approach on other NLP tasks 
such as the argument reasoning comprehension task~\cite{habernal2017arg} where reliable
classifications may be difficult to obtain.
We also plan to investigate whether the GP preference function can be trained 
using a combination of classifications and absolute scores as well as pairwise labels.
%The idea of combining scores and pairwise labels 
%using an active learning approach for image quality assessment with crowds was demonstrated by
%\cite{ye2013combining}, albeit using the Laplace approximation for the Gaussian process. 
%In future we plan to investigate the effectiveness of this idea using our 
%scalable Gaussian process approach for interactive learning settings with implicit feedback.
