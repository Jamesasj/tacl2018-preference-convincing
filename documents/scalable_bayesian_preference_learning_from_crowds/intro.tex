\section{Introduction}\label{sec:intro}

Many tasks are more suited to pairwise comparisons than classification etc. 
Crowds of non-expert annotators may label more accurately if presented with pairs. 
Implicit feedback may be taken from user actions in an application that can be represented as a preference, such as choosing
an option over other options.

There are several works for learning from noisy pairwise comparisons so far (Horvitz et al. 2013 or something like that?). 
However, these do not provide a way to take account of item features or to model different but valid subjective viewpoints. 
They assume there is a single ground truth and can therefore model only one task and one user's (or a consensus of all users) preferences at once. 

Work by Felt et al. 2015, Simpson et al. 2015 etc. shows that item features are particularly useful when combining crowdsourced data. A Gaussian process has not been tested for this purpose before?

GP preference learning presents a way to learn from noisy preferences but assumes constant noise and a single underlying preference function. 
The collaborative Gaussian process (Houlsby et al. 2012) learns multiple users' preferences. 
However existing implementations do not scale and do not identify ground truth. 

We show how to scale it using SVI and how to use the model to identify ground truth from subjective preferences. 

In this paper, we develop methodology to solve the following questions:
\begin{enumerate}
  \item How can we learn a rating function over large sets of items given a large number of pairwise comparisons?
  \item How do we account for the different personal preferences of annotators when inferring the ground truth?
\end{enumerate}
To answer these questions we make the following technical contributions:
\begin{enumerate}
 \item We propose a method for predicting either gold-standard or personalized ratings by aggregating crowdsourced preference labels using a model of the noise and biases of individual annotators. % say why this is not possible with Dawid and Skene
  \item %To enable collaborative preference learning % collaborative preference learning needs to be defined above
   To enable this method to scale to large, real-world datasets, we develop
   stochastic variational inference for Bayesian matrix factorization and Gaussian process preference learning.
  \item To expedite hyper-parameter tuning, we introduce a technique for gradient-based length-scale optimization of Gaussian processes.
\end{enumerate}
