\RequirePackage[fleqn]{amsmath}
\RequirePackage{fix-cm}
%
%\documentclass{svjour3}                     % onecolumn (standard format)
%\documentclass[smallcondensed]{svjour3}     % onecolumn (ditto)
\documentclass[smallextended,natbib]{svjour3}       % onecolumn (second format)
%\documentclass[twocolumn]{svjour3}          % twocolumn
%
\smartqed  % flush right qed marks, e.g. at end of proof
%
\usepackage{graphicx}

%\usepackage{times}
%\usepackage{latexsym}
%\usepackage{multirow}
\usepackage{url}
\makeatletter
\makeatother
%\usepackage[hidelinks]{hyperref}
%\DeclareMathOperator*{\argmax}{arg\,max}

% \usepackage{times}
% \usepackage{url}
% \usepackage{latexsym}
% 
% \usepackage[fleqn]{amsmath}
% \usepackage{amssymb}
% \usepackage{amstext}
% \usepackage{amsthm}
% 
%\usepackage{cite}

\usepackage{amsfonts}
\usepackage{algorithm2e}
\usepackage{array}
\usepackage[caption=false,font=footnotesize]{subfig}
\usepackage{url}
\usepackage{tabularx}
\usepackage{numprint}
\usepackage{multirow}

\newcommand{\bs}{\boldsymbol}  
\newcommand{\wrtd}{\mathrm{d}}

\makeatletter
\makeatother %some sort of hack related to the symbol @

%%%%%%%%%%%%%%%%%%%%%%%%%%%%%%%%%%%%%%%%%%%%%%%%%%%%%%%%%%%%%%%%%

\title{ 
Scalable and Sparse: Bayesian Preference Learning with Crowds
}

\author{Edwin Simpson 
\and Iryna Gurevych \\
}
\institute{Ubiquitous Knowledge Processing Lab, Dept. of Computer Science, Technische Universit\"at Darmstadt, Germany\\
              \email{\{simpson,gurevych\}@ukp.informatik.tu-darmstadt.de}
}
\date{Received: date}
\begin{document}

% do not exceed 20 pages including references

\maketitle

\begin{abstract}
We show how to make collaborative preference learning work at scale and how it can be used to learn
a target preference function from crowdsourced data or other noisy preference labels. 
The collaborative model captures the reliability of each worker or data source and models their biases and error rates. 
It uses latent factors to share information between similar workers and a target preference function.
We devise an SVI inference schema to enable the model to scale to real-world datasets.
Experiments compare results using standard variational inference, laplace approximation and SVI.
On real-world data we show the benefit of the personalised model over a GP preference learning approach 
that treats all labels as coming from the same source,
as well as established alternative methods and classifier baselines.
We show that the model is able to identify a number of latent features for the workers and for textual arguments.
\end{abstract}

% For peer review papers, you can put extra information on the cover
% page as needed:
% \ifCLASSOPTIONpeerreview
% \begin{center} \bfseries EDICS Category: 3-BBND \end{center}
% \fi
%
% For peerreview papers, this IEEEtran command inserts a page break and
% creates the second title. It will be ignored for other modes.
%\IEEEpeerreviewmaketitle

%%%%%%%%%%%%%%%%%%%%%%%%%%%%%%%%%%%%%%%%%%%%%%%%%%%%%%%%%%%%%%%%%

\section{Introduction}\label{sec:intro}

We hypothesise that different people find different types of argument more convincing 
than others and therefore, 
textual features have varying levels of importance in determining convincingness, 
depending on the audience. 
We investigate whether certain combinations of textual features are indicative of an argument's convincingness to a particular person.
We hypothesise that predictions of convincingness will be more accurate if we adapt the model to the individual reader based on their previously observed preferences. 
However, preference data for a single individual for any given task can be very sparse, 
so it will be necessary to consider the similarities between different users' preferences.
Furthermore, the  computational cost of learning independent models for each person and each task may be
impractically high, suggesting a need for more efficient approaches that combine information from multiple users.

Our approach is therefore to identify correlations between different people's preferences
so that we can learn shared models of convincingness that can then be adapted to individuals to improve predictions of argument convincingness. 
We aim to establish whether such a model can be learned by observing pairwise convincingness preferences, 
%and whether language features extracted from arguments can further improve performance when which argument a person will find most convincing. 

The experiments evaluate a number of techniques for modelling worker preferences, different types of language features, and the correlations between workers and features. 
We investigate whether workers with similar preferences according to each model give similar justifications for their decisions, thereby lending additional support for models based on correlations between preferences.

We provide a new preference learning model to handle large numbers of potentially very sparse features and large numbers of people. Our Bayesian approach enables us to 
perform automatic feature selection, learn in semi-supervised or unsupervised modes, 
and fully account for model and parameter uncertainty, while scaling to large numbers of input features. 

\section{Related Work}\label{sec:related}

The Gaussian process (GP) preference learning approach of \cite{chu2005preference} resolves such inconsistencies and provides a way to predict rankings or preferences for 
items for which we have not observed any pairwise comparisons based on the item's features. 
An extension to multiple users was proposed by \cite{houlsby2012collaborative}, 
but this method suffered from poor scalability.

Matrix factorisation techniques are commonly used in recommender systems to discover latent
user and item features but can fail if the
data is very sparse unless suitably regularised or given a Bayesian treatment.
Matrix factorisation techniques are also unsuitable for pairwise comparisons as they 
must be learned using explicit numerical ratings.
A more scalable approach that incorporates probabilistic matrix factorisation
(specifically, probabilistic PCA) was proposed by \cite{khan2014scalable}.
Their method is applicable to both pairwise comparisons and ratings data
and as such could be used to learn the model from implicit feedback such as clicks on an item. However, it may be more suitable to use a model for such feedback that explicitly considers the different bias and noise of each type or source of feedback. For such
a purpose, the model of \cite{dawid_maximum_1979} may be appropriate but has to date
been used for classifier combination and categorical labelling tasks in crowdsourcing and has not been applied to preference learning from different types of feedback. 
Bayesian approaches are suited to handling these problems of data sparsity, noise and bias, 
particularly as the modular nature of inference algorithms such as Gibb's sampling and variational approximation is suited to extending the model to handle different types of feedback that give indications of some underlying preferences. 

The GP methods require $\mathcal{O}(P_n)$ steps, where $P_n$ is the number of pairs for 
user $n$. 
The method proposed by \cite{khan2014scalable} reduces this scaling issue by using a random sample of pairs at each iteration of their EM algorithm.
We use SVI to address scalability in a variational Bayesian framework. 
The modular nature of VB allows us to take advantage of models for feedback of different types
where the input values for each type of feedback do not directly correspond (e.g. explicit user ratings and number of clicks may have different values).
By using SVI, we provide a formal way to deal with scalability that comes with guarantees\cite{hoffman2013stochastic}.
We also estimate the output scale of the GPs, the latent factors, and item bias as part of the 
variational approximation. %not clear what the true advantage of this is?

We compare our work on Sushi-A dataset or against the method of \cite{khan2014scalable} to see if 
our modifications are actually useful. 

Factor analysis differs from PPCA in allowing only diagonal noise covariance matrices, making 
the observed variables conditionally independent given the latent variables. It also provides
a probabilistic treatment for inferring the latent features. %are we still using FA?

We also investigate whether argumentation preferences can be reduced to a simpler
clustering structure, which may be easier to learn with very sparse user data.

% possible extension: state variable to describe what was previously seen? This could relate to time 
% since argument seen, and can be converted to an input feature for the GP model: exp(-t). I think
% that learning length scale and output scale for this feature would work.

%%%%% New additions in March 2017 -- edited a little when moved from other paper in August

In most scenarios where we wish to make predictions about arguments, 
there is a very large number of input variables potentially associated with each argument in the dataset,
but very sparse observations of these variables. 
To illustrate this, consider a simple bag-of-words representation of the argument text, and a set
of click-data recording which actions each user took when presented with a choice between different pieces of text. 
Given a large vocabulary, the words present in an argument will be a very small subset of possible words. Users will likely see a subset of texts and the recorded choices will be a much smaller subset of 
the possible combinations of texts. 
To make predictions about unobserved preferences when presented with a new text with sparse data,
we require an abstraction from the raw input data, and thus seek a way to embed the texts into a space 
where texts with similar properties are placed close together. In the case of arguments, one property
that may determine whether texts should be close together is that they have similar levels of 
convincingness to similar types of people, in similar contexts. Our proposal therefore produces
a form of argument embedding, driven by convincingness.
%Other work on argument embeddings was carried out by \cite{???}. 
A similar approach to learning latent features, VBMDS, is proposed by \cite{soh2016distance} for learning embeddings using approximate Bayesian techniques, but does not use the embeddings for 
preference learning to find separate person and item embeddings and does not apply this to NLP problems.
Their proposal does, however, show how to combine points with and without side information -- our
input features -- to make predictions about low-dimensional embeddings for unseen data. 
The kernelized probabilistic matrix factorization (KPMF) \cite{zhou2012kernelized} 
proposes a similar approach to VBMDS using GP priors over latent dimensions, but with a simpler
MAP inference scheme, and different likelihood and distance functions. 
% see section 4.1 in soh2016distance for more related work in this area, such as GPLVM.

An important aspect of convincingness is the context in which an argument is made, particularly
as part of a dialogue. 
In our approach, this context can be represented as input variables that affect the item and person embeddings, where the variables encapsulate the previously seen arguments.
While out-of-scope of the present investigation, future work may investigate the best way to
determine novelty of an argument given a small number of variables representing previously seen arguments.
Another related avenue of improvement is to consider the structure of arguments to select 
argument components -- it may be important to consider not just novelty, but whether claims have 
sufficient support and premises are clearly linked to the claims they support or attack. 
Embedding this structure may require complex graph structures of claims and premises to be represented
as short vectors, and may therefore be a topic of future study. 


% % I think this is future work?
% Secondly, kernel functions are not typically learned
% or adapted to the data, which means that points with different features that commonly co-occur are
% not assigned high covariance, whereas it would be desirable to learn that commonly co-occurring features
% indicate similar target values. 
% A solution to this problem is to represent input features such as words using vectors of continuous values, i.e. word embeddings. This approach was proposed for performing GP regression on 
% text data by \cite{yoshikawa2015non}, who showed how to learn the word embeddings and map document
% distributions over word embeddings to points in a reproducing kernel Hilbert space. 
% % this is what we need for using probabilistic embeddings? Do current probabilistic/Gaussian embeddings
% % just try to infer expected embedding and use it as input to another method? If so, 
% % we could see if there is an improvement in using kernel embeddings of distributions. The kernel
% % embedding is quite simple actually -- just the expectation of the kernel value with respect to the 
% % uncertain variable. The challenge would be to turn this into point value that can be used as 
% % input to a NN that uses no explicit kernel function.... or do they do something equivalent?
% This approach can be used to obtain document embeddings from word embeddings.

% E.g. product review texts. Training data contains +ve reviews with word "good". Unlabelled data
%contains reviews where "good" and "excellent" co-occur --> generative model learns to associate 
%"excellent" with +ve reviews. A GP regression model with "good" and "excellent" as binary input features
% would not be able to learn to associate "excellent" with +ve reviews through co-occurrence, it would 
% rely on "good" being present. 


The latent features allow us to interpolate between items and people in a low-dimensional embedding space. 
A key question in this latent feature approach is how to model the deviation of individual 
preferences from that predicted by latent features common to multiple people (item deviations
can be modelled through an item mean function).
This deviation occurs when there is still entropy in a user's preferences given the latent features
because the latent features only describe patterns that are common to multiple users.
A simple approach is to allow additional noise with uniform variance at each data point, 
so that all preference patterns are represented by the latent feature vectors of items and people.
However, any individual preference patterns particular to one user must then be represented by additional
latent features that are not activated for any other users. 
An alternative is to use a personal model of preference deviation for each person. 
Given the input features of the items and any state variables relating to the person, 
this model can capture correlations in the deviation for different items for the same person. 
Both the latent person features and the individual noise model can also include any input features of 
the person that change over time, e.g. representing their state and the arguments they have 
previously seen. 
This individual noise model allows us to differentiate preference patterns that are specific to 
one user, when the input features may not otherwise be sufficient to distinguish these users. 

\section{Related Work}
\label{sec:rw}

\subsection{Preference Learning from Crowds}

Several works have analyzed bounds on error rates or sample complexity for pairwise 
learning~\citep{chen2015spectral,shah2015estimation}, but do not propose methods
for learning multiple rankings from crowds of users.
% Most approaches use Bradley-Terry or Thurstone. However some also try Mallows models ~\cite
% {busa2014preference,raman2015bayesian} to get the uncertainty over the ranking instead of over a 
% latent score.
% Other approaches use graph-based ranking measures, e.g. based on Kleinberg's HITS ~\citep{sunahase2017pairwise} or PageRank.
% extension of chenc2013 to k-ary preferences. han2018robust
% wang2016blind/Thurstonian Pairwise Preference (TPP): Chen et al. lacks the mechanism to model multiple query domains, (what does this mean? --> better at labeling certain types of items)
%thus incapable to characterize workers’ domain-dependent expertise and truthfulness. 
% CROWD BT does not take query difficulty into account either. (do they have a feature-dependent model?)
% Furthermore, unlike TPP, CROWD BT does not model the generation of rankings (it models generation of pairwise labels directly)
% Therefore, it simplifies the generation of inconsistent annotations as solely a result from worker accuracy
% (it treats differences between workers as pure noise).
~\citet{chen2013pairwise} account for the varying quality of pairwise labels obtained from a crowd
by learning an individual model of agreement with the true pairwise labels for each worker.  
This approach treats the inconsistencies between annotators' labels as noise and does not consider the
items' features. Therefore, this method does not learn the workers' individual preferences
and cannot model how their accuracy depends on the items considered. 
%Say we want to learn a 'ground truth' preference function that may be one user's preference function.
% One set of pairwise labels may be informative for one subset of items.
% E.g. music recommendation, two users may have similar jazz preferences but all other genres are different
% E.g. learning user preferences from webpage clicks, selecting items from a list may be informative in one context and meaningless in another
In contrast, ~\citet{fu2016robust} consider item features when learning to rank from pairwise labels, 
but do not model individual annotators at all.
~\citet{uchida2017entity} do model the confidence of individual annotations
and propose a fuzzy ranking SVM to make predictions given item features. 
However, their approach also assumes a single ranking over items.
The benefit of jointly learning to rank and group items has also been explored\citep{li2018simultaneous}, again assuming a single ordering.

~\citet{tian2012learning} consider crowdsourcing tasks where there may be more than one correct answer.
They use a nonparametric Dirichlet process model to infer a variable number of clusters of answers for each task,
and also infer annotator reliability. 
However, they do not apply the approach to ranking using pairwise labels.
Several other works learn multiple rankings from crowdsourced pairwise labels
rather than a single gold-standard ranking, 
but do not consider the item or user features so cannot extrapolate to new users or 
items~\citep{yi_inferring_2013,kim2014latent,wang2016blind,kim2017latent}. 
Both \citet{yi_inferring_2013} and \citet{kim2017latent} learn a small number of
latent ranking functions that can be combined to construct personalized preferences, 
although neither provide a Bayesian treatment to handle data sparsity.
\citet{wang2016blind} consider the case where different rankings correspond to lists of items
provided in response to search queries. 
While they model the dependence of annotator accuracy on the domain of a query,
their approach was not applied to personal or subjective rankings.
%queries belong to 'domains'. Annotators have different accuracy on each domain. 
% we don't do that because we assume each annotator has their own ranking and so different noise levels are introduced through the personalized preference function having larger values where the annotator is confident. 

% Also consider mentioning relevant work on active learning
% -- finding the most preferred item with minimal labels
% -- learning a preference function using AL
% -- learning within a budget constraint
% Strategy: look at recent works from ML/AI/DM conferences. Must consider annotators with different preferences.
A number of studies consider actively selecting pairs of items for
comparison to minimize the number of pairwise labels required~\citep{radlinski2007active,qian2015learning,maystre2017just,cai2017pairwise}.
Related research treats the selection of pairwise labels as a multi-armed bandit problem~\citep{busa2018preference}.
In this work, we do not study the process of learning from an oracle or user that we can query.
Rather, we develop a model for aggregating pairwise labels from multiple sources, 
which can be used as the basis of active learning methods that exploit the model uncertainty 
estimates provided by this Bayesian approach.

\subsection{Bayesian Preference Learning}

%include the work on collaborative GPPL
A Bayesian approach to preference learning with Gaussian processes, \emph{GPPL}, 
uses item features to can make predictions for unseen items and
share information between similar items~\citep{chu2005preference}.
 This model assumes a single preference function over items, so cannot
be used to model the individual preferences of multiple users.
The approach was extended by ~\citet{houlsby2012collaborative}
to capture individual preferences using a latent factor model. 
Pairwise labels from users with common interests help to predict each other's preference function, hence 
this can be seen as a \emph{collaborative} learning method, as used in \emph{recommender systems}.
The inference techniques proposed for this model mean it scales poorly, with computational complexity $\mathcal{O}(N^3 + NP)$, where $N$ is the number of items and $P$ is the number of pairwise labels, and memory complexity $\mathcal{O}(N^2 + NP + P^2)$. In this paper, we address this issue and adapt the model for aggregating crowdsourced data.

%other Bayesian recommender systems that deal with noisy preferences

\subsection{Bayesian Matrix Factorization}
% previous work on how to make BMF more scalable with larger datasets or Latent factor analysis etc.
% who has used GPs for BMF?

Preference data can be represented as an item-user matrix with $N$ rows and $M$ columns, 
where $N$ is the number of items and $M$ is the number of users.
In this paper, we are interested in the task of predicting values in this matrix
given only sparse observations of pairwise comparisons.
Matrix factorization techniques are commonly used to discover latent user and item features but 
can fail if the data is very sparse, unless suitably regularised or given 
a Bayesian treatment~\citep{salakhutdinov2008bayesian}. 
Recent work on scaling Bayesian matrix factorization (BMF) to large datasets has focused on parallelizing 
inference~\citep{ahn2015large,vander2017distributed,chen2018large}. Instead of distributing the computation,
this paper focuses on reducing the computational cost, although the method we propose is amenable
to parallelization.

 % PCA: Gaussian noise. "The classical PCA converts a set of samples with possibly correlated variables into another   
 % set of samples with linearly uncorrelated variables via an orthogonal transformation [1]. Based on this, PCA
 % is an effective technique widely used in performing dimensionality reduction and extracting features." -- Shi et al 2017. shi2017survey
 % SVD: like PCA with the mean vector set to zeroes.
 % variations of PCA: for handling outliers or large sparse errors
 % most matrix factorizations are special cases of PCA and in practice do not consider the mean vector.
 % probabilistic PCA: latent variables are unit isotropic Gaussians --> all have 0 covariance and 1 variance.
 % Bayesian PCA: places priors on all latent variables.
 % Probabilistic factor analysis: assumes different variances on each of the latent factors.
 % Probabilistic matrix factorisation: ignores the mean. --> I.e. can be done with SVD
 % I think this means our method is a form of PFA? But extended to consider correlations in the weights.
 % NMF: as matrix factorisation but the low-rank matrices are non-negative.
Several extensions of BMF use Gaussian process priors over latent factors 
to model correlations between 
items given side information or observed item features~\citep{adams2010incorporating,zhou2012kernelized,houlsby2012collaborative,bolgar2016bayesian}. 
However, these techniques are not directly applicable to 
learning from pairwise comparisons 
as they assume that the observations are Gaussian-distributed numerical ratings~\citep{shi2017survey}. 

To combine Bayesian matrix factorization with a pairwise likelihood,
~\citet{houlsby2012collaborative} propose a combination of expectation propagation and variational Bayesian inference.
However, their proposed method does not scale sufficiently to the numbers of 
items, users or pairwise labels found in many important application domains. 
% of their hybrid inference method expectation propagation and variational Bayes limits its application in many domains.
% They use FITC to provide a sparse approximation. This is still not as scalable as SVI and doesn't work as well --
% see the Hensman papers?
In contrast, ~\citet{khan2014scalable} develop a scalable variational EM algorithm
for matrix factorization but combine this with a separate GP to model each user's preferences. 
%with Gaussian processes to improve performance with sparse data. 
%Their model can be learned using pairwise labels, numerical ratings, or other likelihoods.
However, while the proposed method can be trained with pairwise labels, 
it does not capture correlations between items or users in the latent factors.
%user features nor model dependencies between item features 
%and the low-dimensional latent features, so it cannot exploit the latent features to predict preference scores for new items and relies instead on a user-specific GP.
%In contrast, our approach does not require learning a separate GP per user, but instead
%places GPs on both the latent factors. This means that the item and user
%features assist in learning the latent factors as we can exploit their similarities and
%correlations.
%relying instead on a purely individual GP with no shared information,
%(this would be a problem if the user fits the latent features exactly as the GP will end up modelling the user's individual devaiation from the common preferences modeled by the latent features) 
Furthermore, their scalable inference method sub-samples training data rather than learning from the complete training set.
%To achieve scalability using a variational EM algorithm, ~\citet{khan2014scalable}
%sub-sample the pairwise labels meaning that 
%some training data must be discarded. In this paper, we
%applies stochastic variational inference to learn from all data 
%while limiting memory and computational requirements.
% how do we compare to them or do we get out of it? --> compare results on the same datasets
%Our approach is similar to Khan et al. 2014. "Scalable Collaborative Bayesian Preference Learning" but differs in that
%we also place priors over the weights and model correlations between different items and between different people.
%Our use of priors also encourage sparseness in the features. 
%TODOs:
% what is meant by 'factorization assumptions' exactly and do we make them? I think we do but don't fully
% understand why they're so bad. See [18,11] from Khan for examples of bad factorization. 

\subsection{Stochastic Variational Inference}
% use of SVI for making Bayesian methods more scalable, including GPs
% To the best of our knowledge, SVI has not previously been applied to BMF
Models that combine Gaussian processes with non-Gaussian likelihoods require approximate inference
methods that often scale poorly with the amount of training data available. 
This problem can be tackled using \emph{Stochastic variational inference (SVI)}~\citep{hoffman2013stochastic}. 
SVI has been successfully applied to Gaussian processes~\citep{hensman2013gaussian}, including Gaussian process classifiers~\citep{hensman2015scalable}, and Gaussian process preference learning (GPPL)~\citep{simpson2018finding}.
This paper adapts SVI to Bayesian matrix factorization for the first time as part of a solution for
collaborative preference learning. 
We also provide the first full derivation of SVI for GPPL
and introduce a technique for efficiently tuning the length-scale hyper-parameters of the Gaussian processes.

\section{Bayesian Preference Learning for Crowds}\label{sec:model}

%%%% Notes

% Title or name of the model: 
% -- cannot decide this until we get most of the paper complete: will emphasis be on crowds? distilling ground truth
% from noisy sources (Bayesian preference learning for fusing unreliable sources)? user preferences/collaborative filtering?
% -- need a new name to differentiate from Houlsby et al. and Khan et al.?
% -- what are the differences in the model? Let's get the model written up.
% -- should also be some buzzword or word to generate interest: 
%    -- 'variational' is on the up, could be used in paper title
%    -- 'stochastic variational' is also on the up
%    -- 'crowdsourcing' on the way down as at 2010 level
%    -- 'gaussian process' on the way up
%    -- 'matrix factorization' kind of on the way up
%    -- 'scalable' on the way up
%    -- 'interactive learning' on the way down
%    -- 'preference learning' flattish, may be on way up slowly
% -- aimed at crowdsourcing problems (uses a common mean function as consensus?)
% -- other parameters for importance of features?
% -- combined preference learning? Preference aggregation? Collaborative crowdsourced preference learning? Bayesian preference learning for crowds? another word for 'multi-user' or 'many users and many items' vs. crowds?

% TO ADD: Why does the variance in f cancel out when predicting the probability of a pairwise label?
% TO ADD: Why does \sigma disappear if we learn the output scale.

% We also estimate the output scale of the GPs, the latent factors, and item bias as part of the 
% variational approximation allowing us to estimate these parameters in a Bayesian manner without 
% resorting to maximum likelihood approaches.

% mention how the noise model deals with inconsistencies in preferences

% \begin{enumerate}
% \item What are the benefits of Bayesian methods and Gaussian processes in particular?
% \item The proposal by \citep{chu2005preference} shows how the advantages of a Bayesian
% approach can be exploited for preference learning by modifying the observation model

% Extensions:
% -- how do we replace the GP with a NN?
% -- would this move us from a Bayesian to an ML solution?
% -- maybe save this for future work? Or add a few lines if we can make it fit with the theme of the paper.
% -- another is to replace the fixed number of clusters with a CRP, then the whole thing can be nonparameteric preference learning with crowds.

\subsection{Modeling Pairwise Preferences}

% Include the case for one user -- preferences may depend on a number of observed features.

% this should be introduced in section 1. \citep{handleycomparative} -- compares BT with TM models
A pairwise comparison, $y(a \succ b)$, between items $a$ and $b$ 
has a binary label that is one if $a$ is preferred to $b$, 
or zero if $b$ is preferred to $a$ (also written $a \prec b$).
We assume that the likelihood of pairwise label $y(a,b)$ depends on the underlying
value of the items to the user, represented through a latent function of the items' features,
$f(\bs x_a)$, where $\bs x_b$ is a vector representation of the features of item $a$.
The relationship between the value function, $f$, and the pairwise labels
can be modeled by any of several different likelihood functions,
including the Bradley-Terry model~\citep{bradley1952rank,plackett1975analysis,luce1959possible}
and the Thurstone-Mosteller model~\citep{thurstone1927law,mosteller2006remarks}.
The Bradley-Terry model takes the following form:
\begin{align}
p(y(a \succ b) | f) & = \frac{1}{1 + \exp( f(\bs x_a) - f(\bs x_b) ) }
\end{align}
This is a logistic likelihood, which allows pairwise labels that do not reflect 
the true relative values of the items, due to labeling errors, variability in the user's judgements, or if
the preferences are derived from noisy implicit data such as clicks streams.
The error rate is determined by the relative difference in $f$ values of the items.

A different view is to treat the errors as the result of random noise in the value function:
\begin{align}
 p(y(a \succ b) | f, \delta_{a}, \delta_{b} )  
 \hspace{0.9cm} & = \begin{cases}
 1 & \text{if }f(\bs x_a) + \delta_{a} \geq f(b) + \delta_{b} \\
 0 & \text{otherwise,}
 \end{cases} &
 \label{eq:thurstone}
\end{align}
where $\delta \sim \mathcal{N}(0, 0.5)$ is Gaussian-distributed noise.
Integrating out the unknown values of $\delta_{i}$ and $\delta_{j}$,
we get a probit likelihood:
\begin{align}
p( y(a \succ b) | f ) 
& = \int\int p( y(a \succ b) | f, \delta_{a}, \delta_{b} ) \mathcal{N}(\delta_{a}; 0, 0.5)\mathcal{N}(\delta_{b}; 0, 0.5) d\delta_{a} d\delta_{b} &\nonumber\\
& = \Phi\left( z \right), 
\label{eq:plphi}
\end{align}
where $z = f(\bs x_a) - f(\bs x_b)$,
and $\Phi$ is the cumulative distribution function of the standard normal distribution. 
This is a Thurstone-Mosteller model, sometimes referred to as \emph{Thurstone case V}, and was used for Gaussian process preference learning (GPPL) 
by \citet{chu2005preference}, with the difference that they learned the variance of the random noise, $\delta$
rather than assuming it is $0.5$. However, this is unnecessary in practice, since we 
scale instead the value function, $f$, to reduce or increase the certainty in the pairwise labels.
Both the logistic and probit approaches can be used here, but we proceed with
the probit likelihood (as in \citep{herbrich2007trueskill,chu2005preference}
because it allows us to handle uncertainty in $f$ in a simple manner %as Gaussian noise
by modifying $z$:
% we haven't mentioned that f is Gaussian yet. 
\begin{align}
\hat{z} = \frac{\mu_a - \mu_b}{\sqrt{1 + \sigma_a + \sigma_b - \sigma_{a,b}} }
\end{align}
where $\mu_a$ and $\mu_b$ are the expected values of $f(\bs x_a)$ and $f(\bs x_b)$ respectively, 
$\sigma_a$ and $\sigma_b$ are the corresponding variances,
and $\sigma_{a,b}$ is the covariance between $f(\bs x_a)$ and $f(\bs x_b)$.

\subsection{Single User Preference Learning}

First consider modeling the preferences of a single user. In this case, we assume that the value function, 
$f$, is a function of item features and has a Gaussian process prior: 
$f \sim \mathcal{GP}(0, k_{\theta}/s)$, where $k_{\theta}$ is a kernel function with hyper-parameters $\theta$, 
and $1/s$ is the scale of the function
drawn from a gamma prior, $s \sim \mathcal{G}(\alpha_0, \beta_0)$, with shape $\alpha_0$ and scale $\beta_0$.
The value of $s$ determines the variance of $f$ and therefore its magnitude, which affects the level of certainty
in the pairwise label likelihood, Equation \ref{eq:plphi}.
The kernel function takes item features as inputs and determines the covariance between values of $f$ for different items. Typically, we choose a kernel function that produces higher covariance between items with similar feature values,
such as the \emph{squared exponential} or \emph{Mat\'ern} functions.
The choice of kernel function is a model selection problem as it controls the shape and smoothness of the function 
across the feature space. However, the Mat\'ern and squared exponential make minimal assumptions and so are effective in a wide range of tasks (see ~\citet{rasmussen_gaussian_2006} for more). 

% Covariance of different items with same feature: worth mentioning here? We need to multiply the kernel by a small
% amount < 1 to ensure covariance is not 1 and the values can differ.

We observe a set of $P$ pairwise preference labels for a single user, $\bs y=\{y_1,...,y_P\}$,
where the $p$th label, $y_p=y(a_p \succ b_p)$.% refers to items $\{ a_p, b_p \}$.
The joint distribution over all variables is as follows:
\begin{flalign}
p\left( \bs{y}, \bs f, s | k_{\theta}, \alpha_0, \beta_0 \right) 
=  \prod_{p=1}^P p( y_p | \bs f ) 
\mathcal{N}(\bs f; \bs 0, \bs K_{\theta}/s) \mathcal{G}(s; \alpha_0, \beta_0) \nonumber \\
=  \prod_{p=1}^P \Phi\left( z_p \right) 
\mathcal{N}(\bs f; \bs 0, \bs K_{\theta}/s) \mathcal{G}(s; \alpha_0, \beta_0), &
\label{eq:joint_single}
\end{flalign}
where $\bs f = \{f(\bs {x}_1),...,f(\bs {x}_N)\}$
are the latent values for the $N$ items referred to by the pairwise labels 
$\bs y$, and $\theta$, $\alpha_0$ and $\beta_0$ are hyper-parameters.

\subsection{Latent Factors: Bayesian Matrix Factorization}

% 0. Consider multiple users, each with a set of observed features.
% 1. imagine augmenting the observed features with a number of latent features (this is kind of what Khan et al. do)
% 2. imagine that the latent features relate preference function values between items and users using the smallest number of features
% 3. hence the observed features are not needed given the latent features, but we can create a hierarchical model where the latent features depend on the observed ones if available.

We wish to exploit similarities between the value functions of different users or label sources to improve our preference model, particularly when faced with sparse data.
% is there a better word than 'label sources' for the different sources of implicit feedback or other types of labeling?
In a scenario with multiple users or label sources, 
we can represent preference values in a matrix, $\bs{F}$,
where rows correspond to items, columns to users, and entries are preference values.
If we factorize this matrix, we obtain two lower-dimensional matrices,
one for users, $\bs{W} \in \mathcal{R}^{D \times U}$, 
and one for the items, $\bs{V} \in \mathcal{R}^{N \times D}$,
where $D$ is the latent dimensionality, $U$ is the number of users, and $N$ is
the number of items: $\bs{F} = \bs{V}^T \bs{W}$.
Each row of $V$ matrices is a vector representation of an item, 
while each row of $W$ is a vector representation of a user, 
both containing the values of latent features.
Users with similar values for a certain feature will have similar preferences for 
the subset of items with corresponding feature values. 
The features could represent, for example, in the case of book recommendation, interests in a particular genre of book. 
Using vector representations for users and items reflects that users may have overlapping sets of interests,
and that items may have multiple features that make them attractive.

Besides latent features, we may also observe a number of item features, $\bs x$,
and user features, $\bs u$. 
%There are two ways that observed features can be incorporated into the 
%model: (1) as additional dimensions of V or W (each feature cannot be a member of %both); (2) as input features on which V and W depend. The advantage of the latter is that 
In the single user model, we assumed a single latent value function, $f$, of the observed item features. 
For the multi-user case, we assume that there are $D$ latent functions, $v_d$ over 
item features and $D$ latent functions, $w_d$, over user features,
and thereby model the relationship between each observed feature and each of the latent features. 
The matrices $V$ and $W$ are evaluations of these functions at the points corresponding to 
the observed users and items.
Therefore, the latent preference function, $f$, for a user with features $\bs u$ is 
a weighted sum over latent functions:
\begin{align}
  f(\bs x_a, \bs u_j) = \sum_{d=1}^D w_d(\bs u_j) v_d(\bs x_a)
\end{align}
%differently for each user and item. For example, the observed user feature 'age' may correlate with some latent interests of users, but certain users will deviate from their peer group. 
% what happens if two users have identical features (say, the feature representation
% has only simple values, such as age in years)? They have 1-1 covariance, but there 
% is variance in the GP at one location, so both can be drawn separately from the prior.
To provide a Bayesian treatment to matrix factorization, we place Gaussian process priors over the latent functions:
\begin{align}
v_d \sim \mathcal{GP}(\bs 0, k_{\theta} /s_d) & & w_d \sim \mathcal{GP}(\bs 0, k_{\theta}).
\end{align}
It is not necessary to learn a separate scale for $w_d$, since $v_d$ and $w_d$ are multiplied with each other, making a single $s_d$ equivalent to the product of two separate scales. 
The choice of $D$ can be treated as a hyperparameter, or modeled using a non-parametric prior, such as 
the Indian Buffet Process, which assumes an infinite number of latent factors ~\citep{ding2010nonparametric}.
For simplicity, we assume fixed values of $D$ in this paper, and allow the scale parameter $s_d \approx 0$ 
to effectively remove any dimensions that are not required to model the data.
This section described a Bayesian matrix factorization model, 
which we will subsequently extend to a preference learning model for crowds of users and label sources. 

%TODO: include a mention of nonparametric IBP priors over infinite D. 

\subsection{Crowd Preference Learning} \label{sec:crowd_model}

% joint distribution
% notes about problems with inference.

We combine the matrix factorization method with the preference likelihood of Equation \ref{eq:plphi}
to obtain a joint preference model for multiple users or label sources.
In addition to the latent factors, we introduce a common value function over item features, 
$t\sim \mathcal{GP}(\bs 0, k_{\theta} /\sigma_t)$, 
that is shared across all users. 
Its values $\bs t = \{t(\bs {x}_1),...,t(\bs {x}_N)\}$ represent a consensus between users,
if present, while allowing individual users' preferences to deviate from this value through $\bs V^T \bs W$. 
Hence, $\bs t$ can model the underlying ground truth in crowdsourcing scenario, or when using
multiple label sources to learn preferences for one individual.
The joint distribution of this crowd model is:
\begin{flalign}
p\left( \bs{y}, \bs V, \bs W, \bs t, s_1, ..., s_D, \sigma_t | k_{\theta}, \alpha_0, \beta_0 \right) 
= & \prod_{p=1}^P \Phi\left( z_p \right) 
\mathcal{N}(\bs t; \bs 0, \bs K_{t,\theta} /\sigma_t)
\mathcal{G}({\sigma_t}; \alpha_0, \beta_0) \nonumber \\
& \hspace{-2.6cm} \prod_{d=1}^D \left\{
\mathcal{N}(\bs v_d; \bs 0, \bs K_{v,\theta} /s_d) 
\mathcal{N}(\bs w_d; \bs 0, \bs K_{w,\theta}) \mathcal{G}(s_d; \alpha_0, \beta_0) \right\}, \\
\textrm{where } z_p = &  \bs v_{.,a_p}^T \bs{w}_{.,j_p} - \bs v_{.,b_p}^T \bs{w}_{.,j_p}, &
\label{eq:joint_crowd}
\end{flalign}
and $\sigma_t$ is the inverse scale of $t$.
The index $p$ now refers to a tuple, $\{j_p, a_p, b_p \}$ that identifies the user and a pair of items.

\section{Scalable Inference}\label{sec:inf}

Given a set of pairwise labels, $\bs y$, 
the goal is to infer the posterior distribution over the preference values $\bs f$, in the single
user case, or $\bs F=\bs V^T \bs W$ in the multi-user case. 
Previous approaches include a Laplace approximation for the single user case~\citep{chu2005preference}
and a combination of expectation propagation (EP) with variational Bayes (VB) for a 
multi-user model~\citep{houlsby2012collaborative}.
The Laplace approximation is a maximum a-posteriori (MAP) solution that
takes the most probable values of parameters rather than integrating over their distributions
and has been shown to perform poorly for tasks such as classification~\citep{nickisch2008approximations}. 
EP approximates the true posterior with a simpler, factorized distribution
that can be learned using an iterative algorithm.
The true posterior is multi-modal, since the latent factors can be re-ordered arbitrarily without
affecting $\bs F$: this is the non-identifiability problem.
A standard EP approximation would average these modes before predicting $\bs F$,
producing uninformative predictions over $\bs F$.
~\citet{houlsby2012collaborative} resolve this by incorporating a VB step, which approximates a single mode.
A drawback of EP is that unlike VB, convergence is not guaranteed~\citep{minka2001expectation}.
%do they also linearise in the same way? -- both linearise. But EP uses a joint over y and f as its approximation to p(y|f), then optimises the parameters iteratively. It's not guaranteed to converge. Variational EGP instead approximates
% p(y|f) directly with the best fit Gaussian. It's not clear whether this could be updated iteratively but it doesn't
% seem to work if done simultaneously with the other variables we need to learn (the linearisation), 
% perhaps because the algorithm for learning the weights breaks if the variance of q(y|f), Q, keeps changing. 
% Possibly because Q does not change incrementally. So it's
% possible that an outer loop could be used.

The cost of inference can be reduced using a \emph{sparse} approximation based on a set of 
\emph{inducing points}, which act as substitutes for the set of points in the training dataset.
By choosing a fixed number of inducing points, $M \ll N$, the computational cost is fixed at $\mathcal{O}(M^3)$.
These points must be selected so as to give a good approximation, 
using either heuristics or optimizing their positions to maximize the approximate
marginal likelihood. 
\citet{houlsby2012collaborative} use a FITC approximation~\citep{snelson2006sparse} 
with their EP method to limit the costs of inference. However, in practice, FITC 
is unsuitable for datasets with more than a few thousands points
as it is not amenable to distributed computation, does it address other expensive operations
with computational complexity $\mathcal{O}(NP)$ and memory complexity $\mathcal{O}(P^2 + NP + N^2)$, which may 
become limiting when the number of data points is large\citep{hensman2015scalable}.  
We turn to stochastic variational inference (SVI) ~\citep{hoffman2013stochastic} 
to derive a more scalable approach
for Gaussian process preference learning, including
a multi-user model founded on Bayesian matrix factorization.
% how does hensman 2015 optimize inducing points? This is one selling point of SVI. 
% Advantages of VB:  Nickisch 2008 got poorer results for VB methods they tried than EP. But our method may be different?, Seeger 2000
First, we define an approximate posterior that can be estimated using SVI, 
then provide the update equations for an iterative algorithm to optimize
this approximation. We begin with the model for a single user, then extend this to the multi-user case using matrix factorization. 

\subsection{An Approximate Preference Likelihood}

Due to the non-Gaussian likelihood, Equation \ref{eq:plphi},
the posterior distribution over $\bs f$ contains intractable integrals:
\begin{flalign}
p(\bs f | \bs y, k_{\theta}, \alpha_0, \beta_0) = 
\frac{\int \prod_{p=1}^P \Phi(z_p) \mathcal{N}(\bs f; \bs 0, \bs K_{\theta}/s) 
\mathcal{G}(s; \alpha_0, \beta_0) d_s}{\int \int \prod_{p=1}^P \Phi(z_p) \mathcal{N}(\bs f'; \bs 0, \bs K_{\theta}/s) 
\mathcal{G}(s; \alpha_0, \beta_0) d s d f' }.
\label{eq:post_single}
\end{flalign}
To simplify the integral in the denominator, we approximate the preference likelihood with a Gaussian:
\begin{flalign}
\prod_{p=1}^P \Phi(z_p) \approx \mathcal{N}(\bs y; \Phi(\bs z), \bs Q),
\label{eq:likelihood_approx}
\end{flalign}
where $\bs z=\{z_1, ..., z_P\}$
and $\bs Q$ is a diagonal noise covariance matrix.
We estimate the diagonal entries of $\bs Q$ by moment matching
the approximate likelihood with a beta-binomial with variance given by:
\begin{flalign}
Q_{p,p}=\mathbb{E}_{\bs f}[\Phi(z_p)(1 - \Phi(z_p))] 
%= \mathbb{E}_{\bs f}[\Phi(z_p)] - \mathbb{E}_{\bs f}[\Phi(z_p)]^2 - \mathbb{V}_{\bs f}[\Phi(z_p)], 
= \frac{ (y_p + \gamma_0)(1-y_p + \lambda_0) }{ (2 + \gamma_0 + \lambda_0},
\end{flalign}
where $\gamma_0$ and $\lambda_0$ are parameters of a Bernoulli distribution that has the same variance as the prior $p(\Phi(z_p) | \bs K_{\theta}, \alpha_0, \beta_0)$ using numerical integration.
Setting $\bs Q$ in this way matches the moments of the true likelihood, $\Phi(z_p)$,
to those of the Gaussian approximation.

Unfortunately, the nonlinear term, $\Phi(\bs z)$ means that the posterior is still intractable, 
so we linearize $\Phi(\bs z)$ by taking its first-order Taylor series expansion
about the expected value of $\bs f$:
\begin{flalign}
\Phi(\bs z) &\approx \tilde{\Phi}(\bs z) = \bs G (\bs f-\mathbb{E}[\bs f]) + \Phi(\mathbb{E}[\bs z]), \\
G_{p,i} &= \Phi(\mathbb{E}[z_p])(1 - \Phi(\mathbb{E}[z_p])) (2y_p - 1)( [i = a_p] - [i = b_p]) 
\end{flalign}
where $\bs G$ is the Jacobian matrix of the pairwise likelihood with elements $G_{p,i}$. 
This creates a dependency between the posterior mean of $\bs f$ and the linearization terms in the likelihood,
which can be estimated iteratively using variational inference~\citep{steinberg2014extended},
as we will describe below.
The linearization makes the approximate likelihood conjugate to $\mathcal{N}(\bs f; \bs 0, \bs K_{\theta}/s)$,
so that the approximate posterior over $\bs f$ is also Gaussian. 

Given our likelihood approximation, we can now use variational inference to estimate the marginal over $\bs f$,
by optimizing an approximate posterior over all latent variables:
\begin{flalign}
p(\bs f, s | \bs y, \bs x, k_{\theta}, \alpha_0, \beta_0) & \approx q(s)q(\bs f), & \nonumber \\
\textrm{where } \log q(s) & = \log \mathcal{N}(\mathbb{E}[\bs f]; \bs 0, \bs K_{\theta}/s) + \log \mathcal{G}(s; \alpha_0, \beta_0) + \textrm{const}, & \nonumber \\
\log q(\bs f) & = \log \mathcal{N}(\bs y; \tilde{\Phi}(\bs z), \bs Q) + \log \mathcal{N}(\bs f; \bs 0, \bs K_{\theta}/\mathbb{E}[s]) + \textrm{const}. &
\label{eq:vb_approx}
\end{flalign}
%don't some of the expectations over s simplify out?
The Gaussian likelihood approximation and linearization
also appear in methods based on expectation propagation~\citep{rasmussen_gaussian_2006} 
and the extended Kalman filter~\citep{reece2011determining,steinberg2014extended}.
However, neither these methods nor our approximation in Equation \ref{eq:vb_approx}
make inference sufficiently scalable, as they all require
inversion of an $N \times N$ matrix and further computations involving $N \times P$ and $P \times P$ matrices.
We therefore modify Equation \ref{eq:vb_approx} to enable stochastic variational inference (SVI).

\subsection{Sparse Approximation for the Single-User Model}

We introduce a sparse approximation to the Gaussian process that allows
us to limit the size of the covariance matrix that needs to be inverted,
and permit stochastic inference methods that consider only a subset of the $P$ observations at each iteration
~\citep{hensman2013gaussian,hensman2015scalable}. 
To do this, we introduce a set of $M$ \emph{inducing points}, with inputs $\bs x_m$,
 function values $\bs f_m$, and covariance $\bs K_{mm}$.
The covariance between the observations and the inducing points is $\bs K_{nm}$.
We then modify the variational approximation in Equation \ref{eq:vb_approx} to introduce the inducing points 
(for clarity, we omit $\theta$ from this point on):
\begin{flalign}
p(\bs f, \bs f_m, s | \bs y, \bs x, \bs x_m, k_{\theta}, \alpha_0, \beta_0) &\approx q(\bs f, \bs f_m, s) = q(s)q(\bs f)q(\bs f_m), \label{eq:svi_approx}
\end{flalign}
\begin{flalign}
\log q(\bs f_m) &= \log \mathcal{N}\left(\bs y; \tilde{\Phi}(\bs z), \bs Q\right)]
+ \log\mathcal{N}\left(\bs f_m; \bs 0, \bs K_{mm}/\mathbb{E}\left[s\right]\right)  + \textrm{const}, \nonumber \\
%&= \log \int \mathcal{N}(\bs y - 0.5; \bs G \bs f, \bs Q) 
%\mathcal{N}(\bs f; \bs A \bs f_m, \bs K - \bs A \bs K_{nm}^T) & \nonumber\\
%& \hspace{3.2cm} \mathcal{N}(\bs f_m; \bs 0, \bs K_{mm}\mathbb{E}[1/s]) \textrm{d} \bs f + \textrm{const} & \nonumber\\
 & = \log \mathcal{N}(\bs f_m; \hat{\bs f}_m, \bs S ), \\
\bs S^{-1} &= \bs K^{-1}_{mm}/\mathbb{E}[s] + \bs A^T \bs G^T \bs Q^{-1} \bs G \bs A, \label{eq:S}\\
\hat{\bs f}_m &= \bs S \bs A^T \bs G^T \bs Q^{-1} (\bs y - \Phi(\mathbb{E}[\bs z]) + \bs G \mathbb{E}[\bs f] ), \label{eq:fhat_m}
\end{flalign}
where $\bs A = \bs K_{nm} \bs K^{-1}_{mm}$.
The factor $q(s)$ remains unchanged from Equation \ref{eq:vb_approx}, while
$q(\bs f)$ is now assumed to be independent of the observations:
 \begin{flalign}
\log q(\bs f) & = \log \mathcal{N}(\bs f; \bs A \hat{\bs f}_m, 
\bs K + \bs A (\bs S - \bs K_{mm}/\mathbb{E}[s]) \bs A^T ).
\end{flalign}
The use of inducing points therefore avoids the need to invert an $N \times N$ covariance matrix to compute the posterior.

To choose inducing points that can represent the spread of data in our observations
across feature space, 
we use K-means++~\cite{arthur2007k} with $K=M$ to  
cluster the feature vectors, 
then take the cluster centers as inducing points.
%Compared to standard K-means, K-means++ introduces a new method 
%for choosing the initial cluster seeds that
%provides theoretical bounds on the error function. In practice, this 
%reduces the number of poor-quality clusterings.
An alternative approach would be to learn the placement of inducing points
as part of the variational inference procedure ~\citep{???},
or by maximizing the variational lower bound on the log marginal likelihood 
(see next section).
However, the former breaks the convergence guarantees, and both approaches
may add substantial computational cost. 
Therefore, in this paper, we show that it is often sufficient to place inducing points up-front, and leaving their optimization for future work. 

\subsection{SVI for Single-User Preference Learning}

%The approximate posterior can now be optimized using stochastic variational inference (SVI),
%which uses a series of stochastic updates involving different subsets of the observations.
%Variational inference 
To estimate the approximate posterior, we can apply variational inference, which
iteratively reduces the KL-divergence between our approximate posterior, $q(s)q(\bs f)q(\bs f_m)$
and the true posterior, $p(s, \bs f, \bs f_m | \bs K, \alpha_0, \beta_0, \bs y)$,
by maximizing a lower bound, $\mathcal{L}$, on the marginal likelihood, $\log p(\bs y | \bs K, \alpha_0, \beta_0)$ :
\begin{flalign}
\log p(\bs y | \bs K, \alpha_0, \beta_0) & = \textrm{KL}(q(\bs f, \bs f_m, s)  || p(\bs f, \bs f_m, s | \bs y, \bs K, \alpha_0, \beta_0)) - \mathcal{L}. &
\end{flalign}
By taking expectations with respect to the variational $q$ distributions, the lower bound is:
\begin{flalign}
\mathcal{L} =\; & \mathbb{E}_{q(\bs f, \bs f_m, s)}[\log p(\bs y | \bs f) + \log p(\bs f_m, s | \bs K, 
\alpha_0, \beta_0) -\log q(\bs f_m) - \log q(s) ] & \nonumber \\ \label{eq:lowerbound}
=\; & \sum_{p=1}^P \mathbb{E}_{q(\bs f)}[\log p(y_p | f_{a_p}, f_{b_p})] - \frac{1}{2} \bigg\{ \log|\bs S| - M + \log|\bs K_{mm}| - \mathbb{E}[\log s] 
\nonumber &\\
& + \hat{\bs f}_m\mathbb{E}[s] \bs K_{mm}^{-1}\hat{\bs f}_m + 
\textrm{Tr}(\mathbb{E}[s] \bs K_{mm}^{-1} \bs S) \bigg\}  + \log\Gamma(\alpha) - \log\Gamma(\alpha_0)  + \alpha_0(\log \beta_0) \nonumber\\
& + (\alpha_0-\alpha)\mathbb{E}[\log s]+ (\beta-\beta_0) \mathbb{E}[s] - \alpha \log \beta, &
\end{flalign}
%         invK_mm_expecFF = self.invK_mm.dot(self.uS + self.um_minus_mu0.dot(self.um_minus_mu0.T))
%         self.rate_s = self.rate_s0 + 0.5 * np.trace(invK_mm_expecFF)
where $\alpha= \alpha_0 + \frac{M}{2}$ and $\beta = \beta_0 + \frac{
\textrm{Tr}(\bs K^{-1}_{mm}(S + \hat{\bs f}_m \hat{\bs f}_m^T))}{2}$,
and the terms relating to $\mathbb{E}[p(\bs f | \bs f_m) - q(\bs f)]$ cancel.
The iterative algorithm proceeds by updating each of the $q$ factors in turn,
taking expectations with respect to the other factors. 

The only term in $\mathcal{L}$ that refers to the observations, $\bs y$, 
is a sum of $P$ terms, each of which refers to one observation only.
This means that $\mathcal{L}$ can be maximized iteratively by considering a random subset of 
observations at each iteration~\citep{hensman2013gaussian}.
Hence, we replace Equations \ref{eq:fhat_m} and \ref{eq:S} for computing
$\hat{\bs f}_m$ and $\bs S$ over all observations with a sequence of stochastic updates.

For the $i$th update, we randomly select observations $\bs y_i = \{ y_p \forall p \in D_i \}$.
Rather than using the complete matrices, we perform updates using subsets:
$\bs Q_i$ contains rows and columns for observations in $\bs D_i$,
$\bs K_{im}$ and $\bs A_i$ contain only rows referred to by $y_p \forall p \in \bs D_i$,
$\bs G_i$ contains rows in $\bs D_i$ and columns referred to by $y_p \forall p \in \bs D_i$,
and $\hat{\bs z}_i = \{ \mathbb{E}[\bs z_p] \forall p \in D_i \}$.
%All matrices with subscript $_i$ contain only the subset of elements relating to 
%observations in $\bs D_i$.
% The linearization matrix $\bs G_i$ is the subset of elements in $\bs G$ relating to observations in $\bs D_i$, 
%  is the corresponding subset of elements in $\bs Q$,
%  is the covariance between the items referred to by pairs in $\bs D_i$ 
% and the inducing points,
% and  contains the corresponding rows of $\bs A$.
The update equations optimize the natural parameters of the Gaussian distribution by following the
natural gradient~\citep{hensman2015scalable}:
\begin{flalign}
\bs S^{-1}_i  & = (1 - \rho_i) \bs S^{-1}_{i-1} + \rho_i\left( \mathbb{E}[s]\bs K_{mm}^{-1} + w_i\bs K_{mm}^{-1}\bs K_{im}^T \bs G^T_{i} \bs Q^{-1}_i \bs G_{i} \bs K_{im} \bs K_{mm}^{-T} \right)& 
\label{eq:S_stochastic} \\
\hat{\bs f}_{m,i}  & = \bs S_i \left( (1 - \rho_i) \bs S^{-1}_{i-1} \hat{\bs f}_{m,i-1}  + 
\right. \nonumber \\
& \hspace{1.5cm} \rho_i w_i 
\bs K_{mm}^{-1} 
\left.\bs K_{im}^T \bs G_{i}^T \bs Q_i^{-1} \left( \bs y_i  - \Phi(\mathbb{E}[\bs z_i]) + \bs G_{i} \bs A_i \hat{\bs f}_{m,i} \right) \right) & 
\label{eq:fhat_stochastic}
\end{flalign}
where
$\rho_i=(i + delay)^{-forgettingRate}$ is a mixing coefficient that controls the update rate,
$w_i = \frac{P}{|D_i|}$ weights each update according so sample size,
and $delay$ and $forgettingRate$ are hyperparameters of the algorithm~\citep{hoffman2013stochastic}, .


The scale parameter, $s$, can also be learned as part of the SVI procedure. Its variational factor,
$q(s)$, has the following update equations:
\begin{flalign}
\mathbb{E}[s] & = \frac{2a_0 + M}{2b} \label{eq:Es}\\
\mathbb{E}[\log s] & = \Psi(2a_0 + M) - \log(2b), \label{eq:Elogs}
\end{flalign}
where $\Psi$ is the digamma function.

The complete SVI algorithm is summarized in Algorithm \ref{al:singleuser}.
\begin{algorithm}
 \KwIn{ Pairwise labels, $\bs y$, training item features, $\bs x$, 
 test item features $\bs x^*$}
 \nl Compute kernel matrices $\bs K$, $\bs K_{mm}$ and $\bs K_{nm}$ given $\bs x$
 \nl Initialise $\mathbb{E}[s]$, $\mathbb{E}[\bs f]$and $\hat{\bs f}_m$ to prior means
 and $\bs S$ to prior covariance $\bs K_mm$\;
 \While{$\mathcal{L}$ not converged}
 {
 \nl Select random sample, $\bs D_i$, of $P$ observations
 \While{$\bs G_i$ not converged}
  {
  \nl Compute $\bs G_i$ given $\mathbb{E}[\bs f_i]$ \;
  \nl Compute $\hat{\bs f}_{m,i}$ and $\bs S_{i}$ \;
  \nl Compute $\mathbb{E}[\bs f_i]$ \;
  }
 \nl Update $q(s)$ and compute $\mathbb{E}[s]$ and $\mathbb{E}[\log s]$\;
 }
\nl Compute kernel matrices for test items, $\bs K_{**}$ and $\bs K_{*m}$, given $\bs x^*$ \;
\nl Use converged values of $\mathbb{E}[\bs f]$and $\hat{\bs f}_m$ to estimate
posterior over $\bs f^*$ at test points \;
\KwOut{ Posterior mean of the test values, $\mathbb{E}[\bs f^*]$ and covariance, $\bs C^*$ }
\caption{The SVI algorithm for preference learning with a single user.}
\label{al:singleuser}
\end{algorithm}
The use of an inner loop to learn $\bs G_i$ avoids the need to store the complete matrix, 
$\bs G$.
The inferred distribution over the inducing points can be used 
for predicting the values of test items, $f(\bs x^*)$:
\begin{flalign}
\bs f^* &= \bs K_{*m} \bs K_{mm}^{-1} \hat{\bs f}_m, \\
\bs C^* &= \bs K_{**} + \bs K_{*m} \bs K_{mm}^{-1} (\bs S - \bs K_{mm} / \mathbb{E}[s] ) \bs K_{*m}^T \bs K_{mm}^{-1},
\end{flalign}
where $\bs C^*$ is the posterior covariance of the test items, $\bs K_{**}$ is their prior covariance, and
$\bs K_{*m}$ is the covariance between test and inducing points.
It is possible to recover the lower bound proposed by 
\citet{hensman2015scalable} for classification by generalizing the
likelihood to arbitrary nonlinear functions, and omitting terms relating to $p(s|\alpha_0,\beta_0$ and $q(s)$.
However, our approach avoids expensive quadrature methods by linearizing the likelihood to enable analytical updates. We also infer $s$ in a Bayesian manner, 
rather than treating as a hyper-parameter, which is important for preference learning where $s$ controls the noise level of the observations relative to  $f$. 

\subsection{SVI for Crowd Preference Learning}

We now extend the SVI method to the crowd preference learning model proposed in
Section \ref{sec:crowd_model}.
To begin with, we extend the variational posterior in Equation \ref{eq:svi_approx}
to approximate the crowd model defined in Equation \ref{eq:joint_crowd}.
\begin{flalign}
& p( \bs V, \bs V_m, \bs W, \bs W_m, \bs t, \bs t_m, s_1, ..., s_D, \sigma | \bs y, \bs x, \bs x_m, \bs u, \bs u_m, k, \alpha_0, \beta_0 ) \approx & \nonumber \\
& \hspace{3.2cm} q(\bs V)q(\bs W)q(\bs t)q(\bs V_m)q(\bs W_m)q(\bs t_m)\prod_{d=1}^{D}q(s_d)q(\sigma), &
\end{flalign}
where $\bs u_m$ are the feature vectors of inducing points for the users.
This approximation factorizes the joint distribution between the latent item factors, $\bs V$, the latent user factors, $\bs W$, and the common means, $\bs t$, 
but does not requiring factorization across the latent dimensions $\bs w_1,...,\bs w_D$ and $\bs v_1,...,\bs v_D$.
The variational factors for the inducing points can be obtained by deriving expectations as follows, beginning with the latent item factors:
%TODO rename D_i
%TODO make the vectors all lower case for V_d and W_d
%TODO do something to make sure all the block diags are properly indiciated including A_v... but actually I think they are useless because the off-diagonal blocks only affect the collapsed posterior covariance and are never used in computing any other variables.
%TODO intialization of the factors
%TODO put definition of f in somewhere in the model section 
%TODO big Sigma is variance of W because it's different...
%TODO why is the scaling by W nice and simple without off-diagonals, but scaling by V is not? I think that when Q is diagonal, off-diagonals in any scaling factors are irrelevant.
\begin{flalign}
\log q(\bs V_m) = \;\;& \mathbb{E}_{q(\bs W),q(\bs t)}[\log \mathcal{N}\left( \bs y; \tilde{\Phi}(\bs z), Q \right)] & \nonumber \\
& + \sum_{d=1}^D \log\mathcal{N}(\bs v_{m,d}; \bs 0, \bs K_{v,mm}/\mathbb{E}[s_d]) 
+ \textrm{const} & \nonumber \\
% are the dimensions collapsed to a single MVN?
= \;& \sum_{d=1}^D \log \mathcal{N}(\bs v_{m,d}; \hat{\bs v}_{m,d}, \bs S_{v,d}) & \\
\bs S_{v,d}^{-1} = \;\;& \bs K^{-1}_{v,mm}/\mathbb{E}[s_d] 
+ \bs A_v^T \bs G^T \textrm{diag}(\hat{\bs w}_{d,\bs j}^2 + \bs\Sigma_{d,\bs j,\bs j})\bs Q^{-1} \bs G \bs A_v & \\
\hat{\bs v}_{m,d} = \;\;& \bs S_{v,d} \bs A_v^T \bs G^T \textrm{diag}(\hat{\bs w}_{d,\bs j}) \bs Q^{-1}(\bs y - \Phi\left(\mathbb{E}[\bs z]) + \bs G(\hat{\bs v}_d^T \hat{\bs w}_d)\right), &
\end{flalign}
where $\bs A_v = \bs K_{v,nm} \bs K_{v,mm}^{-1}$
and $\hat{\bs w}_{d}$ and $\bs\Sigma_{d}$ are the posterior mean and covariance of the $d$th latent user factor
and the subscript $._{\bs j} = \{ ._{j_p} \forall p \in 1,...,P \}$ contains rows that correspond to the 
users indicated in the observations, $\bs y$. 
The matrix $\textrm{diag}(\hat{\bs W}_{d,\bs j}^2 + \bs\Sigma_{d,\bs j})$ therefore
scales the diagonal observation precision, $\bs Q^{-1}$, by the latent user factors.
The variational factor for the inducing points of the common item mean follows a similar pattern:
\begin{flalign}
\log q(\bs t_m) = \;\;& \mathbb{E}_{q(\bs V)q(\bs W)}[\log \mathcal{N}\left( \bs y; \tilde{\Phi}(\bs z), Q \right)] 
+ \mathbb{E}[\log\mathcal{N}(\bs t_m; \bs 0, \bs K_{t,mm}/s)] 
+ \textrm{const} & \nonumber \\
= \;\;& \log \mathcal{N}\left( \bs t; \hat{\bs t}, \bs S_t \right) & \\
\bs S_t^{-1} = \;\;& \bs K^{-1}_{t,mm}/\mathbb{E}[\sigma] 
+ \bs A_t^T \bs G^T \bs Q^{-1} \bs G \bs A_t & \\
\hat{\bs t}_{m} = \;\;& \bs S_{t} \bs A_t^T \bs G^T \bs Q^{-1}
\left(\bs y - \Phi(\mathbb{E}[\bs z]) + \bs G(\hat{\bs t})\right), &
\end{flalign}
where $\bs A_t = \bs K_{t,nm} \bs K_{t,mm}^{-1}$.
Finally, the variational factor for the inducing points of the latent user factors
requires a different linearization term, $\bs H \in P \times U$, to sum over items rather than users:
\begin{flalign}
H_{p,k} = \Phi(\mathbb{E}[z_p])(1 - \Phi(\mathbb{E}[z_p]) (2y_p - 1) [u_p = k] % needs to be added or subtracted depending on a or b
\end{flalign} 
%now multiply by V. What about covariances between v?
The variational factor is then as follows:
\begin{flalign}
\log q(\bs W_m) = \;\;& \mathbb{E}_{q(\bs V)q(\bs t)}[\log \mathcal{N}\left( \bs y; \tilde{\Phi}(\bs z), Q \right)] 
+ \sum_{d=1}^D \mathbb{E}[\log\mathcal{N}(\bs w_d; \bs 0, \bs K_{w,mm})]
+ \textrm{const} & \nonumber \\
= \;\;& \sum_{d=1}^D \log \mathcal{N}\left( \bs w_d; \hat{\bs w}_d, \bs \Sigma \right) & \\
\bs \Sigma^{-1}_{d} = \;\;& \bs K^{-1}_{w,mm}
+ \bs A_w^T \left( \bs H^T \textrm{diag}(\hat{\bs v}_{d,\bs a}^2 + \bs S_{d,\bs a, \bs a}) \bs Q^{-1} \bs H^T \right. & \nonumber \\
& \left. + \bs H^T \textrm{diag}(\hat{\bs v}_{d,\bs b}^2 + \bs S_{d,\bs b, \bs b}) 
   \bs Q^{-1} \bs H^T \right) \bs A_w & \\
\hat{\bs w}_{m,d} = \;\;& \bs \Sigma_{d} \bs A_w^T 
\left( \bs H^T\textrm{diag}(\hat{\bs v}_{d,\bs a}) - \bs H^T\textrm{diag}(\hat{\bs v}_{d,\bs b}) \right) \bs Q^{-1} & \nonumber \\
& \left(\bs y - \Phi(\mathbb{E}[\bs z]) + \bs G(\hat{\bs v}_d^T \hat{\bs w}_d)\right), &
\end{flalign}
where the subscripts $._{\bs a} = \{ ._{a_p} \forall p \in 1,...,P \}$
and  $._{\bs b} = \{ ._{b_p} \forall p \in 1,...,P \}$ are lists of indices to the first and 
second items in the pairs, respectively, and $\bs A_w = \bs K_{w,um} \bs K_{w,mm}^{-1}$.

The equations for the means and covariances 
can be adapted for stochastic updating by applying weighted sums over
the stochastic update and the previous values in the 
same way as  Equation \ref{eq:S_stochastic} and \ref{eq:fhat_stochastic}.
The stochastic updates for the inducing points of the latent factors depend 
on expectations with respect to the observed points. 
As with the single user case, the variational factors at the observed items are independent of the observations given the variational factors of the inducing points
(likewise for the observed users):
\begin{flalign}
\log q(\bs V) & = \sum_{d=1}^D \log \mathcal{N}\left( \bs v_d; \bs A_v\hat{\bs v}_{m,d}, 
\frac{\bs K_{v}}{\mathbb{E}[s_d]} + \bs A_v (\bs S_{m,d} - \frac{\bs K_{v,mm}}{\mathbb{E}[s_d]})\bs A_v \right) & \label{eq:qv} \\
\log q(\bs t) & = \log \mathcal{N}\left( \bs t; \bs A_t \hat{\bs t}_m, 
\frac{\bs K_{t}}{\mathbb{E}[\sigma]} + \bs A_t (\bs S_t - \frac{\bs K_{t,mm}}{\mathbb{E}[\sigma]})\bs A_t \right)  & \label{eq:qt}\\
\log q(\bs W) & = \sum_{d=1}^D \log \mathcal{N}\left( \bs w_d; \bs A_w \hat{\bs w}_{m,d}, \bs K_{w} + \bs A_w (\bs\Sigma - \bs K_{w,mm}) \bs A_w \right). &
\label{eq:qw}
\end{flalign}
The expectations for the inverse scales, $s_1,...,s_d$ and $\sigma$, can be 
computed using the formulas in Equations \ref{eq:Es} and \ref{eq:Elogs} by
substituting in the corresponding terms for each $\bs v_d$ or $\bs t$ instead of $\bs f$. 
Predictions in the latent factor model can be made using Equations \ref{eq:qv}, \ref{eq:qt} and \ref{eq:qw} by substituting the covariance
terms relating to observation items, $\bs x$, and users, $\bs u$, with corresponding
covariance terms for the prediction items and users.

As with the single user model, the lower bound on the marginal likelihood 
contains sums over the observations, hence is suitable for stochastic variational
updates:
% APPENDIXIFY
\begin{flalign}
\mathcal{L} & = \label{eq:lowerbound_crowd}
\sum_{p=1}^P \mathbb{E}_{q(\bs f)}[\log p(y_p | \bs v_{a_p}^T \bs w_{a_p} + t_{a_p}, \bs v_{b_p}^T \bs w_{b_p} + t_{b_p})] 
- \frac{1}{2} 
\Bigg\{  \sum_{d=1}^D \bigg\{  - M_n - M_u & \nonumber \\
&  
 + \log|\bs K_{v,mm}| + \log|\bs K_{w,mm}|
\log|\bs S_{v,d}|  - \mathbb{E}[\log s_d] 
+ \hat{\bs v}_{m,d}^T \mathbb{E}[s_d]\bs K_{v,mm}^{-1}\hat{\bs v}_{m,d} & \nonumber \\
& 
+ \textrm{Tr}(\mathbb{E}[s_d] \bs K_{v,mm}^{-1} \bs S_{v,d}) 
+ \log|\bs \Sigma_{d}|  + \hat{\bs w}_{m,d}^T \bs K_{w,mm}^{-1}\hat{\bs w}_{m,d} 
+ \textrm{Tr}(\bs K_{w,mm}^{-1} \bs \Sigma_{d})
\bigg\}
& \nonumber \\
&  
- M_n + \log|\bs K_{t,mm}|
+ \log|\bs S_{t}|  - \mathbb{E}[\log \sigma] 
+ \hat{\bs t}^T \mathbb{E}[\sigma] \bs K_{t,mm}^{-1} \hat{\bs t}  &
\nonumber \\
&
+ \textrm{Tr}(\mathbb{E}[\sigma] \bs K_{t,mm}^{-1} \bs S_{t})
\Bigg\} 
- (D+1)(\log\Gamma(\alpha_0)  + \alpha_0(\log \beta_0))
& \nonumber \\
& + \sum_{d=1}^D \bigg\{ 
\log\Gamma(\alpha_d) + (\alpha_0 - \alpha_d)\mathbb{E}[\log s_d]
+ (\beta_d - \beta_0) \mathbb{E}[s_d] - \alpha_d \log \beta_d \bigg\}
 & 
\nonumber \\ 
& + \log\Gamma(\alpha_{\sigma}) + (\alpha_0 - \alpha_{\sigma})\mathbb{E}[\log \sigma]
+ (\beta_{\sigma} - \beta_0) \mathbb{E}[s_d] - \alpha_{\sigma} \log \beta_{\sigma}
, &
\end{flalign}

In this section, we proposed an SVI scheme for Bayesian matrix factorization given pairwise observations. The inference scheme can readily be adapted to regression or classification tasks by swapping out the preference likelihood, resulting in 
different values for $\bs G$ and $\bs H$. We now show how to learn the  
length-scale parameter required to compute covariances using typical kernel functions, then demonstrate how our method can be applied to learning user preferences or
consensus opinion when faced with disagreement.
 

\section{Gradient-based Length-scale Optimization}\label{sec:ls}

In the previous sections, we defined preference learning models that 
incorporate GP priors over the latent functions.
The covariances of these GPs are defined by a kernel function $k$, 
typically of the following form:
\begin{flalign}
k_{\bs \theta}(\bs x, \bs x') = \prod_{f=1}^F k_f\left(\frac{|x_f - x_f'|}{l_f}, \bs\theta_f \right)
\textrm{, where } 
\bs \theta = \{l_1,...,l_F, \bs\theta_1,...,\bs \theta_F \}
\label{eq:kernel}
\end{flalign}
where $F$ is the number of features, 
$l_f$ is a length-scale hyper-parameter,
and $\bs \theta_f$ are additional hyper-parameters for an individual 
feature kernel, $k_f$.
Each $k_f$ is a function of the distance between the $f$th feature values in 
feature vectors $\bs x$ and  and $\bs x'$.
The product over features in $k$ means that data points have 
high covariance only if the kernel functions, $k_f$, for all features are high 
(a soft AND operator). 
It is possible to replace the product with a sum, causing covariance to increase
for every $k_f$ that is similar (a soft OR operator),
or other combinations of the individual feature kernels.
The choice of combination over features is therefore an additional hyper-parameter.
% citations? 

The length-scale, $l_f$, controls the smoothness of the function, $k_f$,
across the feature space
and the contribution of each feature to the model. 
If a feature has a large length-scale,
its values, $\bs x$, have less effect on $k_{\bs\theta}(\bs x, \bs x') $
than if it has a shorter length-scale.
Hence, it is important to set $l_f$ to correctly capture feature relevance.
A computationally frugal option is the median heuristic: 
\begin{flalign}
 l_{f,MH} = F \mathrm{median}( \{ |x_{i,f} - x_{j,f}| \forall i=1,..,N, \forall j=1,...,N\} ).
\end{flalign}
The motivation is that the median will normalize the feature, so that features
are equally weighted regardless of their scaling. By using a median to perform this 
normalization, extreme values remain outliers with relatively large distances. 
Multiplying the median by the number of features, $F$,
prevents  the average covariance $k_{\bs \theta}(\bs x, \bs x')$ between items
from increasing as we add more features using the 
product kernel in Equation \ref{eq:kernel}.
This heuristic has been shown to work reasonably well for the task of 
comparing distributions~\citep{gretton2012optimal}, but is a simple heursitic
with no guarantees of optimality. 

An alternative method for setting $l_f$ is Bayesian model selection using 
the type II maximum likelihood method, 
which chooses the value of $l_f$ that 
maximizes the marginal likelihood, $p(\bs y | \bs \theta)$.
Since the marginal likelihoods for our models are intractable, we maximize
the variational lower bound, $\mathcal{L}$, as an approximation (
defined in Equation \ref{eq:lowerbound} for a single user, and Equation \ref{eq:lowerbound_crowd} for the crowd model). 
Optimizing a kernel length-scale in this manner is known as automatic relevance determination (ARD)~\citep{rasmussen_gaussian_2006}, since the optimal
value of $l_f$ depends on the relevance of $f$.

% Removing irrelevant features could improve performance, 
% since it reduces the dimensionality of the space of the preference function.
%A problem when using text data is that large vocabulary sizes and additional linguistic features 
%lead to a large number of dimensions, $D$. 
%The standard maximum likelihood II optimisation requires 
%$\mathcal{O}(D)$ operations to tune each length-scale.
To optimize the length-scales efficiently, we turn to gradient-based methods
 such as L-BFGS-B~\citep{zhu1997algorithm}, which allow us to optimize
 all length-scales simultaneously, rather than one-by-one.
 For the single user model, the required gradient of $\mathcal{L}(q)$
(Equation \ref{eq:lowerbound}) with respect to $l_f$ is as follows:
%Following the derivations in Appendix \ref{sec:vb_eqns}, Equation \ref{eq:gradient_ls},
\begin{flalign}
& \nabla_{l_{\! f}} \mathcal{L} =  
\frac{\partial \mathcal{L}}{\partial \hat{\bs f}_m} \frac{\partial \hat{\bs f}_m}{\partial l_f}
+ \frac{\partial \mathcal{L}}{\partial \bs S} \frac{\partial \bs S}{\partial l_f}
+ \frac{\partial \mathcal{L}}{\partial a} \frac{\partial a}{\partial l_f}
+ \frac{\partial \mathcal{L}}{\partial b} \frac{\partial b}{\partial l_f}
\nonumber \\
 & - \frac{1}{2}\! \bigg \{
 \mathbb{E}[s] \hat{\bs f}_{\! m}^T \bs K_{\! mm}^{-1} 
\frac{\partial \bs K_{\! mm}}{\partial l_f} \bs K_{\! mm}^{-1} \hat{\bs f}_{\! m} 
 + \mathrm{tr}\left(
\mathbb{E}[s]\bs S^T\bs K_{\! mm}^{-1} - \bs I \right)
 \frac{\partial \bs K_{\! mm}}{\partial l_f} \bs K_{\! mm}^{-1}
\! \bigg\}.
\end{flalign}
The first terms in the above equation arise because the parameters of the 
variational parameters depend indirectly on the length-scale. 
However, the partial derivatives of $\mathcal{L}$ with respect to these parameters 
is zero when $\mathcal{L}$ is at a maximum. This occurs when the variational
 inference algorithm has converged, hence these terms can be eliminated if we 
 only compute the gradient $\nabla_{l_{\! f}} \mathcal{L}$ after convergence.
Therefore, we run gradient-based optimization as an outside loop around the 
variational algorithm defined in Section \ref{sec:inf}.
Length-scale optimization begins with an initial guess for $l_f$,
for example, in our experiments, we start with the median heuristic.
Given the current value of $l_f$, the optimizer (e.g. L-BFGS-B)
runs the VB algorithm to convergence, computes 
$\mathcal{L}$ and $\nabla_{l_{\! f}} \mathcal{L}$,
then uses them to propose a new candidate value of $l_f$.
The process repeats until the optimizer converges or reaches a maximum number 
of iterations, and returns the value of $l_f$ that maximized $\mathcal{L}$.

%*** The symbol f should be replaced with \varphi or f? due to clash with function f.

Multi-user model.

Given the kernel function defined in Equation \ref{eq:kernel}, which
contains a product over features,
the partial derivative of the covariance matrix $\bs K_{mm}$ with respect to 
$l_f$ is given by:
\begin{flalign}
\frac{\partial \bs K_{mm}}{\partial l_f} 
& = \frac{\bs K_{mm}}{k_{f}(|\bs x_{mm,f}, \bs x'_{mm,f}) }
\frac{ \bs k_{l_f}(|\bs x_{mm,f}, \bs x'_{mm,f})}{\partial l_f} \nonumber ,\\
\end{flalign}

The choice of kernel function...
In our implementation, we choose the Mat\`ern $\frac{3}{2}$ kernel function for $k$
due to its general properties of smoothness
~\citep{rasmussen_gaussian_2006}... add in citation that shows its good performance.
 
 For the Mat\`ern $\frac{3}{2}$ kernel, the 
 $\frac{\partial \bs K_{l_d}}{\partial l_d}$ is a matrix, where each 
entry, $i,j$,  is defined by:
\begin{flalign}
& \frac{\partial K_{d,ij}}{\partial l_d} = 
\frac{3\bs |\bs x_{i,d} - \bs x_{j,d}|^2}{l_d^3} \exp\left( - \frac{\sqrt{3} \bs |\bs x_{i,d} - \bs x_{j,d}|}{l_d} \right). &
\label{eq:kernel_der}
\end{flalign}
% is defined by Equation \ref{eq:kernel_der}.
% Since we cannot compute $\bs K$ in high dimensions, in practice we substitute $\bs K_{mm}$ for $\bs K$,
% $\bs S$ for $\bs C$, $\hat{\bs f}_{m}$ for $\hat{\bs f}$ and $\bs\mu_{m}$ for $\bs\mu$ so that 

We can define an optimization procedure for the length-scales...
By following the gradients of the length-scale given by 

\section{Experiments with Synthetic Data}\label{sec:synth}
\section{Experiments with Real Data}\label{sec:real}

List of experiments to include:
\begin{enumerate}
\item Performance, computation time, memory vs no. inducing points
\item Performance, computation time, memory vs update size
\item Performance, computation time vs different initialisation methods for the inducing points; include different initialisations of K-means
\end{enumerate}

\section{Future Work}

The collaborative preference model can be adapted so that it can be trained using 
classification data, scores/ratings (a regression task), or a mixture of different 
observation types by applying a different likelihood. 
The core of the method is the abstraction of a latent function over items and people, 
dependent on latent features of items and people, with the ability to include side information 
and observed features.
Future work will therefore investigate the ability to learn from multiple types of labelled data, 
(rather than only using preference pairs).

A further direction for future work is to apply this model to transfer learning: 
instead of modelling different latent functions per person, we model latent functions per task. 
Tasks for which the target function follows a similar pattern would then share information in 
a collaborative manner, so that training data for one task can inform similar tasks. This may be 
useful when data is limited, e.g. when performing domain adaptation. In the latter case, there would
need to be sufficient similarity between the features of the texts that are being classified for 
the collaborative effect to take place. 
For example, in argument mining, we may have several training datasets from different topics,
which can be used to learn a model of argument convincingness. Applying a collaborative model would
identify topics with common latent features, which would inform predictions on the target domain 
in parts of the feature space with no training data.


%%%%%%%%%%%%%%%%%%%%%%%%%%%%%%%%%%%%%%%%%%%%%%%%%%%%%%%%%%%%%%%%%%%%%%%%%%%%%%%%

% use section* for acknowledgment
\section*{Acknowledgments}

\cleardoublepage

\bibliographystyle{spbasic}
\bibliography{simpson_scalable_bayesian_pref_learning_from_crowds}

%%%%%%%%%%%%%%%%%%%%%%%%%%%%%%%%%%%%%%%%%%%%%%%%%%%%%%%%%%%%%%%%%%%%%%%%%%%%%%%%%

\end{document}
