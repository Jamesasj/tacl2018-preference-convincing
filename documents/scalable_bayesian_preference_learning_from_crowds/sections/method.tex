\section{Scalable Bayesian Preference Learning}\label{sec:model}

%%%% Notes

% TO ADD: Why does the variance in f cancel out when predicting the probability of a pairwise label?
% TO ADD: Why does \sigma disappear if we learn the output scale.

% We also estimate the output scale of the GPs, the latent factors, and item bias as part of the 
% variational approximation allowing us to estimate these parameters in a Bayesian manner without 
% resorting to maximum likelihood approaches.

% mention how the noise model deals with inconsistencies in preferences

% \begin{enumerate}
% \item What are the benefits of Bayesian methods and Gaussian processes in particular?
% \item The proposal by \cite{chu2005preference} shows how the advantages of a Bayesian
% approach can be exploited for preference learning by modifying the observation model

%%%% The preference likelihood model

Following Chu and Ghahramani~\shortcite{chu2005preference}, 
we model the relationship between a latent preference function, $f$,
and each observed pairwise label, $v_k \succ u_k$, where $k$ is an index into a list of 
$P$ pairs, as follows:
\begin{flalign}
& p( v_k \succ u_k | f(v_k), f(u_k), \delta_{v_k}, \delta_{u_k} ) & \nonumber\\
& \hspace{1.5cm} = \begin{cases}
 1 & \text{if }f(v_k) + \delta_{v_k} \geq f(u_k) + \delta_{u_k} \\
 0 & \text{otherwise,}
 \end{cases} &
\end{flalign}
where $\delta_i \sim \mathcal{N}(0, 1)$ is Gaussian-distributed noise. 
The noise term allows for variations in the observed preferences, which may occur if 
different annotators disagree or change their minds, or if
the preferences are derived from noisy implicit data such as clicks streams.
We deviate from Chu and Ghahramani~\shortcite{chu2005preference} by assuming $\delta_i$
has variance $\sigma=1$, and instead scale the function $f$ relative to this. This formulation
is equivalent but is more convenient for variational inference. 
We marginalise the noise terms to obtain the preference likelihood:
\begin{flalign}
& p( v_k \succ u_k | f(v_k), f(u_k) ) = \Phi\left( \frac{f(v_k) - f(u_k)}{\sqrt 2} \right), & \label{eq:pl}
\end{flalign}
where $\Phi$ is the cumulative distribution function of a standard Gaussian distribution.
For the latent function , $f$, we assume a Gaussian process prior: $f \sim \mathcal{GP}(0, k_{\theta}/s)$, where 
$k_{\theta}$ is a kernel function with hyper-parameters $\theta$, 
and $s \sim \mathcal{G}(a_0, b_0)$ is an inverse scale parameter %controlling the noise level the level of noise in the function and has 
drawn from a gamma prior with shape $a_0$ and scale $b_0$.
The kernel function controls the correlation between values of $f$ at different points in the feature space.

The inference goal is to learn the posterior distribution over the function values $f(i)$ for each item $i$.
Chu and Ghahramani~\shortcite{chu2005preference}
used gradient descent to optimise a Laplace approximation. 
However, this approach produces a maximum a-posteriori (MAP) approximation, 
which takes the most probable values of parameters rather than integrating over their distributions
in a Bayesian manner and has been shown to perform poorly for tasks such as classification \cite{nickisch2008approximations}. 
%Furthermore, the presented approach is not scalable as it requires
%$\mathcal{O}(n^3)$ computation and $\mathcal{O}(n^2)$ memory, 
%where $n$ is the number of observations. 
We address the desire for a better approximation by adapting a variational method based on the extended Kalman filter (EKF) \cite{reece2011determining,steinberg2014extended} 
to the preference likelihood given by Equation \ref{eq:pl}.
%We refer to the set of pairwise labels for the $k$th pair of items as $\bs y_k$.
%To permit inference using the variational method, our model approximates
To make inference tractable, we
approximate the preference likelihood using a Gaussian distribution:
$p( v_k \succ u_k | f(v_k), f(u_k) ) \approx \mathcal{N}( v_k \succ u_k; \Phi(\hat{f}_{v_k} - \hat{f}_{u_k}/\sqrt 2), \nu_{k})$. 
The variance $v_k$ is estimated by moment matching with 
the variance of a beta posterior distribution $\mathcal{B}(p( v_k \succ u_k | f(v_k), f(u_k) ); 1 + [v_k \succ u_k], 2 - v_k \succ u_k)$.
The values of $v_k$ for all pairs form a diagonal matrix $\bs Q$.
%Further details are given in \cite{reece2011determining}. % cite the heatmaps paper
%shouldn't we also explain G? Is this an innovative bit for preference learning?
This approximation means that the posterior distribution over $f$ for items in the training set
is also Gaussian: $p(f(\bs x) | \bs y) \approx \mathcal{N}(f(\bs x) | \hat{\bs f}, \bs C )$
where $\bs x$ is a matrix of input features for the training items. 
Variational inference is then used to optimise the mean $\hat{\bs f}$ and covariance $\bs C$. 
%and the posterior distribution over the inverse scale $s$.
by maximising a lower bound, $\mathcal{L}$, 
on the log marginal likelihood, $p(\bs y | \theta, a_0, b_0)$:
%We derive this bound using Equations \ref{eq:vblb} 
%to \ref{eq:vblb_terms} in Appendix \ref{sec:vb_eqns}, giving the following:
\begin{flalign}
\label{eq:lowerbound}
& \mathcal{L}(q) \approx - \frac{1}{2} \left\{ L \log 2\pi + \log |\bs Q| - \log|\bs C| + \log|\bs K| \right. \nonumber&&\\
& \left. + (\hat{\bs f} - \bs\mu)\bs K^{-1}(\hat{\bs f} - \bs\mu) \right. \nonumber&&\\
& \left. + (\bs y - \Phi(\hat{\bs z}))^T \bs Q^{-1} (\bs y - \Phi(\hat{\bs z}))\right\} \nonumber&&\\
& + \Gamma(a) - \Gamma(a_0) + a_0(\log b_0) + (a_0-a)\hat{\ln s} \nonumber&&\\
& + (b-b_0) \hat{s} - a \log b, &&
\end{flalign}
% should G appear in here as it is also optimised in a variational manner?
where $L$ is the number of observed preference labels, 
$\bs y = \left[v_1 \succ u_1, ..., v_L \succ u_L\right]$ is a vector of binary preference labels,
$\hat{\ln s}$ and $\hat s$ are expected values of $s$,
and $\hat{\bs z} = \left\{ \frac{\hat{f}_{v_k} - \hat{f}_{u_k}}{\sqrt 2} \forall k=1,...,P\right\}$.
%\label{eq:zhat}.
%$\mathbb{E}_q$ is an expectation with respect to the
%variational posterior distribution, $q(\bs f, s)$.
%We apply a variational inference algorithm learn optimal values for the variational parameters 
%$\bs f$, $\bs C$ and $\bs s$ and obtain an approximate posterior over the latent function values $\bs f$.

The variational approach described so far requires a scalable inference algorithm.
We therefore adapt stochastic variational inference (SVI) 
\cite{hensman2013gaussian,hensman_scalable_2015} to preference learning.
For SVI, we assume $M$ \emph{inducing points} with features $\bs x_m$. 
The inducing points act as a substitute for the complete set of feature vectors of the observed arguments,
and allow us to choose $M << N$ to limit the computational
and memory requirements. 
To choose representative inducing points, we use K-means to rapidly cluster the feature vectors, 
then used the cluster centres as inducing points.
Given the inducing points, SVI further limits computational costs by using an iterative algorithm 
that only considers a subset of the data containing $P_n << P$ pairs at each iteration. 
% As with $M$, the value of $P_m$ can be chosen by the developer to fit their hardware requirements. 
% Smaller values will consider less data at each iteration and therefore may require a larger number of iterations to converge.
% However, convergence is hard to predict and depends on the properties of the dataset used.
The algorithm proceeds as follows:
\begin{enumerate}
\item Randomly initialise the mean at the 
inducing points, $\hat{\bs f}_{m}$, the covariance of the inducing points, $\bs S$, 
 the inverse function scale expectations $\hat{s}$ and $\hat{\ln s}$,
 and the Jacobian of the pairwise label probabilities, $\bs G$.
% The latter is required to enable a first order Taylor series approximation, which makes the iterative updates tractable.
\item Select a random subset of $P_n$ pairwise labels.
\item Compute the mixing coefficient, $\rho_i=(n + \mathrm{delay})^{-\mathrm{forgetting_rate}}$, which controls the 
rate of change of the estimates, and the weight $w_n = \frac{P}{P_n}$, which weights each update
 according to the size of the random subsample.
\item Update each variable in turn using equations below.
\item Repeat from step 2 until convergence.
\item Use converged values to make predictions.
\end{enumerate}
The equations for the updates at iteration $n$ are as follows:
\begin{flalign}
%self.invKs_mm.dot(Ks_nm_i.T).dot(self.G.T)
%        Lambda_i = (Lambda_factor1 / Q).dot(Lambda_factor1.T)
& \bs S^{-1}_n  = (1 - \rho_n) \bs S^{-1}_{n-1} + \rho_n\left( \hat{s}\bs K_{mm}^{-1} \right. \nonumber & \\ 
& \left. + w_n\bs K_{mm}^{-1}\bs K_{nm}^T \bs G^T \bs Q^{-1} \bs G \bs K_{nm} \bs K_{mm}^{-T} \right)& 
\label{eq:S} \\
& \hat{\bs f}_{m,n}  = \bs S_n \left( 
(1 - \rho_n) \bs S^{-1}_{n-1} \hat{\bs f_{m,n-1}}  + \rho_n w_n 
\bs K_{mm}^{-1} \right. \nonumber & \\
& \left.\bs K_{nm}^T \bs G^T Q^-1 \left( \frac{1+[v_k\succ u_k]}{3}  - \Phi(\hat{\bs z}_n) - \bs G \hat{\bs f} \right) \right) & \\
%\hat{\bs f}_n & = \bs K_{nm} \bs K_{mm}^{-1} \bs S &\\
%\bs C_n & = \bs K + (\bs K_{*m}\bs K_{mm}^{-1} \bs S - 
%\bs K_{nm}\bs K_{mm}^{-1}) \bs K_{mm}^{-T}\bs K_{nm}^T&\\
&\hat{s} = \frac{2a_0 + N}{2b}
& \\
& \hat{\ln s} = \Psi(2a_0 + N) - \log(2b) & \\
 & \bs G = \frac{1}{ 2\pi}  \exp\left(-\frac{1}{2} \hat{\bs z}_n ^2\right) \label{eq:G}
\end{flalign}
where  $\bs K_{mm}$ is the covariance between values at the inducing points,
$\bs K_{nm}$ is the covariance between the subsample of pairwise labels and the inducing points,
$\hat{\bs z}_n$ is the estimated preference label likelihood for the $n$th subsample,
$b = b_0 + \frac{1}{2} \mathrm{Tr}\left(\bs K_j^{-1}
\left( \bs\Sigma_j + (\hat{\bs f}_j - \bs\mu_{j,i})(\hat{\bs f}_j - \bs\mu_{j,i})^T \right)\right) $, 
and $\Psi$ is the digamma function.
Given the converged estimates, we can make predictions 
for test arguments with feature vectors $\bs x_*$. The posteriors for the latent 
function values $\bs f_*$ at the test points have mean and covariance given by:
\begin{flalign}
\hat{\bs f_*} & =  \bs K_{*m} \bs K_{mm}^{-1} \hat{\bs f}_{m} \\
%         covpair_uS = Ks_nm.dot(self.invKs_mm_uS)
%        fhat = covpair_uS.dot(self.u_invSm) 
%\bs W^*\left(\Phi(\bs \hat{z}) - \sigma(\hat{\bs f}_j) + \bs G(\hat{\bs f}_j - \bs\mu_j )\right) &\\
%             covpair =  Ks_nm.dot(self.invKs_mm)
%            C = Ks_nn + (covpair_uS - covpair.dot(self.Ks_mm)).dot(covpair.T)
\bs C_* & = \frac{\bs K_{**}}{\hat{s}}
+ \bs K_{*m}\bs K_{mm}^{-1} ( \bs S - \bs K_{mm}) \bs K_{mm}^{-T}\bs K_{*m}^T, &
%\bs\Sigma_j^{*} &= \hat{\bs K}^{**}_j -\bs W_j^* \bs G_j \hat{\bs K}^*_j, \label{eq:f_cov}
\end{flalign}
where $\bs K_{*m}$ is the covariance between the test items and the inducing
points.

% We propose a different inference scheme using , which approximates the full parameter distributions
% and allows the developer to reduce the memory requirements to their required level.
% To develop our SVI approach, we adapt a variational inference method  We further adapt this approach to preference learning
%and show how it can be practically applied to NLP problems with large numbers of input features
%by optimising a bound on the marginal likelihood using gradient ascent.

%%%% ARD intro

\subsection{Kernel Length-scale Optimsation}

The prior covariance of the latent function $f$ is defined by a kernel $k$, 
typically of the form $k_{\theta}(\bs x, \bs x') = \prod_{d=1}^D k_d(|x_d - x_d'| / l_d)$, 
where $k_d$ is a function of the distance between the values of feature $d$ 
for item $x$ and $x'$, and a length-scale hyper-parameter, $l_d$, which 
controls the smoothness of the function across the feature space.
The product over features $D$ may be replaced by other combinations, 
such as a sum. 
There are several ways to set $l$, including the median heuristic~\cite{gretton2012optimal}: 
$ l_{d,MH} = \frac{1}{D} \mathrm{median}( \{ |x_{i,d} - x_{j,d}| \forall i=1,..,N, \forall j=1,...,N\} ) $.
We can also optimise $l_d$ by choosing the value that maximises the lower bound on the 
log marginal likelihood, $\mathcal{L}$, defined in Equation \ref{eq:lowerbound}. 
This process is known as maximum likelihood II and is often referred to as automatic relevance determination (ARD)
~\cite{rasmussen_gaussian_2006}, since features with large length-scales are less relevant because their values
have less effect on $k_{\theta}(\bs x, \bs x') $ than features with short length-scales.
% Removing irrelevant features could improve performance, 
% since it reduces the dimensionality of the space of the preference function.
%A problem when using text data is that large vocabulary sizes and additional linguistic features 
%lead to a large number of dimensions, $D$. 
%The standard maximum likelihood II optimisation requires 
%$\mathcal{O}(D)$ operations to tune each length-scale.
The cost of optimisation may be reduced by simultaneously optimising all length-scales 
using a gradient-based method such as L-BFGS-B~\cite{zhu1997algorithm}.
Given our proposed SVI method, we substitute our inducing point approximation into 
Equation \ref{eq:lowerbound} to approximate the gradients of $\mathcal{L}(q)$
with respect to $l_d$ as follows:
%Following the derivations in Appendix \ref{sec:vb_eqns}, Equation \ref{eq:gradient_ls},
\begin{flalign}
& \nabla_{l_d}\mathcal{L}(q) =  
 \frac{1}{2}\hat{s} \hat{\bs f}_{m}^T \bs K_{mm}^{-T} \frac{\partial \bs K_{mm}}{\partial l_d} \bs K_{mm}^{-1} \hat{\bs f}_{m} 
  \nonumber &&\\
& -\frac{1}{2} \mathrm{tr}\left(\left(\hat{s}\bs K_{mm}^{-1}\bs S\right)^{T} 
\left(\bs S^{-1} - \bs K_{mm}^{-1}/\hat{s}\right) \frac{\partial \bs K_{mm}}{\partial l_d}
\right) 
&&.
\end{flalign}
In our implementation, we choose the Mat\`ern $\frac{3}{2}$ kernel function for $k$
due to its general properties of smoothness
~\cite{rasmussen_gaussian_2006}, so that the matrix of partial derivatives is:
\begin{flalign}
\frac{\partial \bs K}{\partial l_d} & = \frac{\bs K}{k_{d}(|\bs x_{d}, \bs x'_{d}) }
\frac{\partial \bs K_{l_d}}{\partial l_d} \nonumber ,\\
\end{flalign}
where each entry $ij$ of the $\frac{\partial \bs K_{l_d}}{\partial l_d}$ is defined as:
\begin{flalign}
& \frac{\partial K_{d,ij}}{\partial l_d} = 
\frac{3\bs |\bs x_{i,d} - \bs x_{j,d}|^2}{l_d^3} \exp\left( - \frac{\sqrt{3} \bs |\bs x_{i,d} - \bs x_{j,d}|}{l_d} \right). &
\label{eq:kernel_der}
\end{flalign}
% is defined by Equation \ref{eq:kernel_der}.
% Since we cannot compute $\bs K$ in high dimensions, in practice we substitute $\bs K_{mm}$ for $\bs K$,
% $\bs S$ for $\bs C$, $\hat{\bs f}_{m}$ for $\hat{\bs f}$ and $\bs\mu_{m}$ for $\bs\mu$ so that 
